% !Tex root = checkedc.tex

\newcommand{\dynamicboundscast}{\texttt{dynamic\_bounds\_cast}}
\newcommand{\dynamicboundscastinst}[2]{\texttt{dynamic\_bounds\_cast<#1>#2}}
\newcommand{\assumeboundscast}{\texttt{assume\_bounds\_cast}}
\newcommand{\assumeboundscastinst}[2]{\texttt{assume\_bounds\_cast<#1>#2}}

\chapter{Interoperation}
\label{chapter:interoperation}

Code that uses checked pointer types must be able to interoperate with
code that uses unchecked pointer types.  This chapter describes support
for this.  Section~\ref{section:pointer-casting} starts with
conversion operations: how different kinds of pointers can
be converted to other kinds of pointers.  
Section~\ref{section:function-bounds-safe-interfaces}
describes how existing code that uses unchecked pointers can be modified
to present a safe interface.  The key insight is that
the interface must be both checked and unchecked, depending on context.
For existing code that uses unchecked pointers, the interface is descriptive,
but correctness is not enforced by the language.  For code that uses checked
pointers, proper usage of the interface is checked and enforced.

\section{Conversions between pointers to objects of different types}
\label{section:pointer-casting}

Conversions from a pointer to one type to a pointer to a different type
introduce two issues. First, there is type safety. Given a pointer to S
that has been converted to be a pointer to T, is it valid to treat the
memory pointed to by the pointer as being an object of type T instead of
type S? Second, there is bounds safety. Given the pointer, what range of
memory can be accessed validly using that pointer? This section focuses on
bounds safety.

Type safety is not addressed by this technical report. Of course, violating
type safety can lead to violations of bounds safety. This can happen
when there is a conversion between a checked pointer to an object of
structure type that contains a bounds-checked member and a pointer
to an object of another type. A programmer can use the pointer to the
other type to modify the bounds-checked member or its bounds in an
inconsistent fashion. For now, it is the programmer's responsibility to
update bounds-checked members and their bounds properly when using a
checked pointer that results from such a conversion. Conversions between
checked pointers to integral types, floating-point types, or structures that
contain only integral types or floating-point types cannot lead to
violations of bounds safety by themselves.

\subsection{Cast operators}
This section describes casts to pointer types.
A conversion of a value from one checked pointer type to a value of another checked
pointer type is valid when the following rules hold:
\begin{itemize}
\item If the destination pointer is a \ptrT\ type, the range of 
memory that can be accessed beginning at the source pointer is
large enough to hold a value of type \var{T}.
\item If the destination pointer is an \arrayptrT\ or \spanptrT\ type,
the bounds for the destination pointer are equal to or within the range of
the memory accessible through the source pointer.
\end{itemize}

Casts between checked pointer types should preserve bounds safety
by default.  The cases where it does not should be the unusual cases.
For this reason, C cast operations to checked pointer types are
required to be provably correct with respect to bounds at compile-time.   It
is straightforward to check casts between checked pointers to constant-sized
objects.   The rules that are used to check bounds and lightweight
invariants are used to check casts as well.   For cases where static
checking cannot prove bounds safety of casts, we add two new operators. 
Table~\ref{table:cast-operators}
describes the operators.

\begin{table}
\begin{tabular}{p{1.5in}p{4in}}
\toprule
Operator & Description \\
\texttt{(T *)} & The C cast operator.  Additional static checking rules
ensure pointer casts to checked pointer types are correct with respect to bounds.\\
\dynamicboundscast\ & Does dynamic checks to ensure bounds
safety.  It produces a runtime error if the checks fail.\\
\assumeboundscast\ & Declares new bounds for the value that are trusted without verification.  
Because it is unchecked, it is not allowed in \keyword{checked} blocks.\\
\bottomrule
\end{tabular}
\caption{Cast operators for pointer types}
\label{table:cast-operators}
\end{table}

The syntax of the new operators is similar to the syntax of C++ type
conversion operators, where the destination pointer type is specified by
placing a type argument \texttt{<\var{T}>} after the operator: \dynamicboundscastinst{\var{T}}{}
and \assumeboundscastinst{\var{T}}{}.   A relative alignment type or constant can be
specified using an optional second argument:  
\dynamicboundscastinst{\var{T}, \var{A}}{} or \assumeboundscastinst{\var{T, \var{A}}}{}.
Of course, macro-like syntax could be used as well.

For C cast operations from checked pointer types to unchecked pointer types,
it is not possible to prove that the uses of the resulting unchecked pointers
are correct at compile-time.   At the same
time, it is desirable to  catch the programming error of converting an out-of-range
checked pointer to an unchecked pointer and then trying to use the unchecked
pointer to access memory.  For this reason, we take the stance that if the
destination pointer is a \uncheckedptrT\ type, the range of memory that can be accessed
beginning at the checked pointer should be provably large enough at compile time
to hold a value of type \var{T}.

C does allow an unchecked pointer to point one element past the end of an array.
However, it is illegal to dereference that pointer.  This case mainly arises
from loops that stride through an array and end up creating a pointer one past
an element of an array.  It would be unusual for a programmer to
convert a checked pointer to an unchecked pointer and then stride downwards through an array,
predecrementing the pointer.  Because of that, C cast operations from checked pointer
types to unchecked pointer types that would create a pointer one element past
the end of the array pointed to by the checked pointer are rejected at compile time.
A programmer may use an \assumeboundscast\ operation do this instead.

\subsection{Description of cast operators}

This section describes each of the cast operators and the static
and runtime checking that is done for each operator.   

For C cast operators, the checking is only static.
Here are the rules for cast operators of the form \texttt{(\var{D}) \var{e}},
where \var{e} has source type \var{S}.  The rules are applied in addition
to any existing C typing rules that apply to the cast.
\begin{itemize}
\item If \var{D} and \var{S} are compatible types or \var{D} and \var{S} are unchecked
pointer types, checking succeeds.

\item Otherwise, if \var{D} is \ptrvoid\ or \uncheckedptrvoid\ and \var{S} is a \ptr\ type,
checking succceeds.\footnote{This rule is implied logically by other rules.   We present
it separately so that there is a simple rule for casts to void pointers.}
\item Otherwise, if \var{D} is not a function pointer type and \var{D} is
\begin{itemize}
\item \ptrT: the bounds of \var{e} are computed using the rules
in Section~\ref{section:checking-nested-assignment-expressions}.
\begin{itemize}
\item If the bounds are \boundsnone, checking fails with a compile-time
error.
\item Otherwise, if the bounds are \boundsany, checking 
succeeds.
\item Otherwise, the bounds must be \bounds{\var{lb}}{\var{ub}} (or convertible
to that form).  It must be provable statically that \texttt{\var{e} >= \var{lb}}
and that {\texttt{((char *) \var{ub}) - ((char *) \var{e}) >= sizeof(\var{T})}}.
If \texttt{sizeof(\var{T})} is not defined because \var{T} is \void, \texttt{1} is
used instead.
\end{itemize}
\item \arrayptrT: No new rules are needed.   The rules in
Chapter~\ref{chapter:checking-bounds} already cover this case.
\item \spanptrT: It must be provable statically that \var{e}
and \bounds{\var{lb}}{\var{ub}} meet the relative alignment requirements of
the destination \spanptr\ type.  At runtime, a new value of
type \spanptrT\ with the value of \var{e} and \bounds{\var{lb}}{\var{ub}} will be created.
\item \texttt{\var{T} *}: If the source type \var{S} is
\begin{itemize}
\item \ptrT, the checking succeeds. This handles the case where \var{T}
is an incomplete type.
\item Otherwise, the rules for \ptrT\ are followed.
\end{itemize}
\end{itemize}

\item Otherwise, if \var{D} is a function pointer type,
\begin{itemize}
\item Function names with unchecked function pointer types can be cast to have corresponding checked
pointer types.  If \var{e} is a function name, \var{D} is \ptrT, \var{S} is \uncheckedptrinst{U},
and \var{T} and \var{U} are compatible types, checking succeeds.
\item Otherwise, checking fails with a compile-time error.
\end{itemize}
\end{itemize}

The \dynamicboundscast\ operator is not strictly needed.
Programmers could write checks by hand.  However, programmers
would have to compute the bounds of expressions by hand too, so it is convenient to
have it. It takes 1 to 3 arguments, depending on the kind of conversion being done:
\begin{itemize}
\item
  \dynamicboundscastinst{\var{T}}{(\var{e1})}
  converts \var{e1} to either a \ptr\ or * type.
\item
  \dynamicboundscastinst{\var{T}}{(\var{e1},
  \var{e2})} converts \var{e1} to an \arrayptr\ or
  \spanptr\ type with bounds \boundscount{\var{e2}}.  
\item
  \dynamicboundscastinst{\var{T}}{(\var{e1},\var{e2},\var{e3})} converts \var{e1} to an
  \arrayptr\ or \spanptr\ type with bounds
  \bounds{\var{e2}}{\var{e3}}.  \dynamicboundscastinst{\var{T}, \var{A}}
  {(\var{e1},\var{e2},\var{e3})}
   optionally changes the relative alignment of the bounds to \var{A}.
\end{itemize}

Here is the checking that is done. The bounds of \var{e1} are computed
using the rules in Section~\ref{section:checking-nested-assignment-expressions}.
If the bounds of \var{e1} are \boundsnone, it is a compile-time error. 
If the bounds of \var{e1} are \boundsany, no runtime checks are needed.
Otherwise the bounds must be \bounds{\var{lb}}{\var{ub}} (or convertible to that form).
The following runtime checks are done:
\begin{itemize}
\item Check whether \var{e1} is 0.  If so, no further runtime checking
is needed.
\item Otherwise, if the operator has the form:
\begin{itemize}
\item
  \dynamicboundscastinst{\var{T}}{(\var{e1})}:
  check that there is room for least one element of \var{T} by doing
  the check for \dynamicboundscastinst{\var{T}}{(e1, 1)}.
\item
  \dynamicboundscastinst{\var{T}}{(\var{e1}, \var{e2})}:
   check that \var{lb}
  \texttt{<=} \var{e1} \texttt{\&\&} \var{e1} \texttt{+
  sizeof(referent-type(}\var{T}\texttt{)) *} e2 \texttt{<=}
  \var{ub}.
\item
  \dynamicboundscastinst{\var{T}}{(\var{e1},\var{e2},\var{e3})}: 
  check that \var{lb} \texttt{<=} \var{e2} \texttt{\&\&}
  \var{e3} \texttt{<=} \var{ub}. Also check that \var{e1}, \var{e2}, and \var{e3} 
  are relatively aligned to \var{T} (or the optional alignment parameter instead, if one
  is given). 

\end{itemize}
\end{itemize}

The operator \assumeboundscastinst{\var{T}}{} declares bounds that are trusted
without verification:
\begin{itemize}
\item
  \assumeboundscastinst{\var{T}}{(\var{e1},\var{e2})}
  converts \var{e1} to an \arrayptr\ or
  \spanptr\ type with \boundscount{\var{e2}}
\item
  \assumeboundscastinst{\var{T}}{(\var{e1},\var{e2},\var{e3})}
  converts \var{e1} to an
  \arrayptr\ or \spanptr\ type with \bounds{\var{e2}}{\var{e3}}.
  It must be statically provable that \var{e1}, \var{e2} and \var{e3}
  are relatively aligned for \var{T} (or the optional
  relative alignment parameter instead, if there is one).
\item
  \assumeboundscastinst{\var{T}}{(\var{e1})}
  converts \var{e1} to a \ptr\ or \texttt{*} type.
\end{itemize}

If any rule depends on \texttt{sizeof(\var{T})} and \var{T} is
an incomplete type, the cast operation that uses the rule shall fail
to check at compile-time.

A subtle point about the C cast operator and \dynamicboundscast\
are that they allow {\em bounds-safe casts} of an expression
\var{e} of unchecked pointer type to a \spanptr\ or
\arrayptr\ type. This is provided that the bounds for \var{e}
can be determined. Only a few kinds of expressions with unchecked pointer
types have bounds that can be determined: address-of expressions
involving variables and uses of array variables.

\subsection{Examples}
\label{section:pointer-cast-examples}

Here are examples of uses of C cast operators to cast unchecked
pointers to checked pointers.  The static checking rules are straightforward
to apply here.  The examples assume that integers are
4 bytes in size:
\begin{verbatim}
int x = 0;
ptr<int> px = (ptr<int>) &x;
array_ptr<char> pax : count(4) = (array_ptr<char>) px;
array_ptr<char> odd_pax : count(3) = (array_ptr<char>) px;

char data[12];
ptr<int> pfirst = (ptr<int>) data; // pointed to 1st element as an integer;
array_ptr<int> pdata : count(3) = (array_ptr<int>) data;

void swizzle(ptr<int> p) {
   array_ptr<char> bytes : count(4) = (array_ptr<char>) p;
   char t0 = bytes[0], t1 = bytes[1], t2 = bytes[2]; t3 = bytes[3];
   bytes[0] = t3, bytes[1] = t2, bytes[2] = t1, bytes[3] = t0;
}
\end{verbatim}

\subsection{Non-examples}

Here are examples of incorrect uses of C cast operators to cast unchecked
pointers to checked pointers.  These examples are rejected because the
source bounds are not large enough to justify the destination bounds or
there are no source bounds.
In this example, \verb|&x| points to only one integer:
\begin{verbatim}
int x = 0;
// fails to check: source not large enough
array_ptr<int> pax : count(5) = (array_ptr<int>) &x;
\end{verbatim}
In this example, the result of \texttt{random()} has no bounds:
\begin{verbatim}
char *random(void);

void f(void) {
    // fails to check: random() has no bounds
    array_ptr<char> sp : count(1) = random();
}
\end{verbatim}

\section{Implicit conversions involving checked pointers}
\label{section:implicit-conversions}

C allows implicit conversions at assignments, function call arguments,
and conditional expressions.  The purpose of
implicit conversions is to make programs shorter and easier to
read.  Implicit conversions describe when an expression of type
\var{S} is allowed where an expression of type \var{T} is expected.
They allow additional programs to typecheck.
This section defines implicit conversions that are allowed for checked
pointer types.

An implicit conversion may allow typechecking to succeed, but
checking of bounds declarations may still fail.  Implicit
conversions are treated as though they are explicit C cast
operations during the checking of bounds declarations.

\subsection{From unchecked pointers to checked pointers}
An expression with an unchecked pointer type can be converted implicitly to an
expression with a checked pointer type.   The destination referent type
and source referent type must be {\em assignment compatible}: they must
either be the same type or the only difference between the two types must be that
the destination type is subject to more checking than the source type.

This can be done for the right-hand side of an assignment, a call argument,
or an arm of a conditional expression.
The type of the left-hand side of the assignment, the parameter, or the other 
arm of the conditional expression must be the checked pointer type that is the
target of the implicit conversion.

A destination referent type \var{D} is assignment compatible with a source referent
type \var{S} if
\begin{itemize}
\item \var{D} and \var{S} are compatible types, or
\item \var{D} is \void\ and \var{S} is not a function type, or
\item \var{D} is an array of \var{E} and \var{S} is an array of \var{T}
and:
\begin{itemize}
\item \var{D} and \var{S} are both checked arrays, both unchecked arrays, or
\var{D} is a checked array and \var{S} is an unchecked array, and
\item \var{D} and \var{S} have the same number of elements, and
\item \var{E} is assignment compatible with \var{T}.
\end{itemize}
\end{itemize}
C allows implicit conversions between \uncheckedptrvoid\ and other pointer
types. For now, implicit conversions from \uncheckedptrvoid\ to checked pointer
types are allowed. This may be revised later when a design
for checking type-safety of casts is complete.

\subsection{From checked pointers to unchecked pointers}

We must be very careful about implicit conversions from checked pointer
types to unchecked pointer types.  These kinds of conversions could allow bounds checking to
be subverted accidentally in a quiet fashion. Implicit conversions from checked pointer types to
unchecked pointer types are allowed only at bounds-safe interfaces
(Section~\ref{section:function-bounds-safe-interfaces}).

\subsection{From checked pointers to checked pointers}

An expression with a checked pointer type can be converted implicitly to the same kind
of checked pointer type with a \texttt{void} referent type.
This can be done for the right-hand side of an assignment, a call argument, 
and an arm of a conditional expression.  
The type of the left-hand side of the assignment, the parameter, or the other
arm of the conditional expression
must be the checked \void\ pointer type that is the target of the implicit
conversion. For example, implicit conversions from 
\ptrinst{\var{T}} to \ptrvoid\ and from \arrayptrinst{\var{T}} to \arrayptrvoid\ are allowed 
at assignments. 

Implicit conversions from checked pointers to \void\ to checked pointers to \var{T} are not allowed.
The philosophy behind this is the same one that is used in C++: places where type-safety
can be compromised by a cast should be explicit in the code.

\subsection{Between checked pointers and integers}

The null pointer (0) can be converted implicitly to any checked pointer type.
A checked pointer can be converted implicitly to the \texttt{\_Bool} type.

Some C compilers extend C by allowing implicit conversions between pointers
and integers or between pointers to incompatible types.  Implicit conversions
from integers to checked pointers are typically not useful in Checked C because
the checking of bounds declarations fails or the resulting pointer cannot
be used to access memory.   The rules for checking bounds declarations only
allow the target type to be \arrayptr\ type and the bounds of the expression to be
\boundsnone.

\subsection{Examples}

The following code shows examples of implicit conversions.  In these examples,
the right-hand sides of assignments have the type ``unchecked pointer to
\var{T}'', while the left-hand sides have the type ``checked pointer to
\var{T}''.  Implicit conversions are done at those assignments.
\begin{verbatim}
int x = 0;
ptr<int> px = &x;
array_ptr<int> pax : count(1) = &x;

int a[5] = { 0, 1, 2, 3, 4};
ptr<int> pa = a;
array_ptr<int> apa : count(5) = a;
array_ptr<int> middle : bounds(a, a + 5) =  a + 2;
\end{verbatim}

An unchecked multidimensional array can be passed to a function expecting
a checked multi-dimensional array:
\begin{verbatim}
int b[3][3] = { { 0, 1, 2}, { 3, 4, 5}, { 6, 7, 8}};
extern void f(int arg checked[3][3]);
f(b);
\end{verbatim}

The mechanics behind this are somewhat complicated.  Parameters with type
``array of \var{T}'' are actually treated as having type ``pointer to \var{T}''.
Parameters with the type ``checked array of \var{T}'' are treated as ``checked
pointer to \var{T}''.  The actual type of the parameter of \texttt{f} is
\begin{verbatim}
extern void f(array_ptr<int checked[3]> arg);
\end{verbatim}
Local variables with type ``array of \var{T}'' are treated as having
type ``unchecked pointer to \var{T}'' when they are used.  The type of
\texttt{b} at \texttt{f(b)} is ``unchecked pointer to an array of 3 integers''.
At the call to  \texttt{f(b)}, there is an implicit conversion from an
``unchecked pointer to an unchecked array of 3 integers'' to a ``checked pointer
to a checked array of 3 integers'':
\begin{verbatim}
int (*pb)[3] = b;
array_ptr<int checked[3]> param : count(3) = pb;
\end{verbatim}
This implicit conversion is allowed because \verb|int checked[3]| is assignment
compatible with \verb|int[3]|.  Assignment compatibility allows other
interesting implicit conversions:
\begin{verbatim}
array_ptr<int[3]> t1 : bounds(b, b + 3)  = b;
array_ptr<int> t2 : bounds(b, b + 3) = t1[1];
array_ptr<int> t3 : bounds(b, b + 3) = b[1];
\end{verbatim}

\subsection{Non-examples}

Implicit conversions must still pass the bounds rules for the corresponding explicit
C cast operation.  Implicit conversions where the source bounds are not large
enough will be rejected during bounds declaration checking:
\begin{verbatim}
int x = 0;
// fails to check: source not large enough
array_ptr<int> pax : count(5) : &x;
\end{verbatim}

Implicit conversions of unchecked pointers with no bounds to checked pointers
will also be rejected:
\begin{verbatim}
char *random(void);

void f(void) {
    // fails to check: random() has no bounds
    array_ptr<char> sp : count(1) = random(); 
}
\end{verbatim}

\section{Bounds-safe interfaces to existing unchecked functions and variables}
\label{section:function-bounds-safe-interfaces}

The new pointer types capture specific properties of pointers. We would like to update
existing C code to use the new pointer types. This would be problematic for library and operating
system APIs that have backward  compatibility constraints, though.   Consider what would happen
if the signature for \texttt{memcpy} were changed to use \arrayptr. The function

\begin{verbatim}
void *memcpy(void *dest, const void *src, size_t count);
\end{verbatim}

becomes

\begin{verbatim}
void *memcpy(array_ptr<void> dest, array_ptr<const void> src, size_t count);
\end{verbatim}

This would break existing code that uses \texttt{memcpy}.  The code would no longer type check.
C does not have function overloading, so we cannot define multiple overloaded versions of
\texttt{memcpy}.  The reverse problem also exists: suppose the signature for \texttt{memcpy} is
not updated. Then every checked method that calls \texttt{memcpy} would need to cast
the arguments to unchecked pointer types.

Given that changing the types of existing APIs is problematic, we take an approach
that does not change types, yet enables new checked to be written easily
and that maintains the checking of new code.  We allow programmers to
declare {\em bounds-safe interfaces} that extend existing declaratons with
additional bounds information.  

Bounds-safe interfaces can be 
specified at declarations for functions, global variables, and
data structure.  They can also be specified for function types.
The interfaces describe the expected behavior and the assumptions
of existing code with respect to bounds information. The types remain the same.
It is assumed but not verified that existing code
meets the specified interfaces.

The additional information is used to ensure that checked code
is using the declarations properly and to insert bounds
checks in checked code.  Type checking is modified to insert
implicit conversions between checked types and unchecked
types at bounds-safe interfaces.  This allows new checked code
to seamlessly use existing unchecked code that has bounds declarations.
The checking of bounds declarations and insertion of bounds
checks also use the additional bounds information:

\begin{enumerate}
\item
  In checked scopes, code is limited to using pointer types that are 
  checked pointer types or unchecked pointer types with bounds-safe interfaces.
  This makes the code in checked scopes straightforward to understand:
  the unchecked pointer types are regarded as checked pointer types, all memory
  accesses are bounds checked or in bounds, and bounds-safe interfaces are trusted
  and respected.
\item
  In unchecked scopes, checked and unchecked pointer types can be
  be intermixed within expressions.  When that happens, the bounds-safe
  interfaces are used during checking of bounds declaration to determine
  bounds of expressions and the required bounds for assignments and function calls. 
  Section~\ref{section:implicit-conversions} already
  explained when implicit conversions from unchecked pointer types to checked pointer
  types may be inserted.   Implicit conversions of expressions 
  from checked pointer types to unchecked pointer types are inserted when
  necessary at bounds-safe interfaces.
\end{enumerate}

\subsection{Specifying bounds-safe interfaces}
Bounds-safe interfaces are specified in two ways.  The first approach
is to use bounds declarations for variables with unchecked pointer types.
The second approach is to use type annotations.

We first describe bounds declarations for unchecked pointer types.  
These can be used at function declarations, function types,
declarations of variables with global scope, and declarations of members
for structure and union types.

Functions that have parameters with unchecked pointer 
types or that return values with unchecked pointer types can have bounds
declared for the parameters and return values.  Implicit
conversions from checked types to unchecked pointer types may
be inserted at calls to these functions.
  
Here is a bounds-safe interface for \texttt{memcpy}:
\begin{verbatim}
void *memcpy(void *dest : byte_count(len), const void *src : byte_count(len), 
             size_t len) 
             where return_value : bounds((char *) dest, (char *) dest + count)
\end{verbatim}
The correctness of bounds information is enforced at compile-time when
\texttt{memcpy} is passed checked pointer arguments. It is not enforced when 
\texttt{memcpy} is passed unchecked pointer arguments.

If bounds behavior is described for one parameter with
unchecked pointer or return value with an unchecked pointer type.
bounds behavior must be described for for all parameters with unchecked
pointer types and the return value, if it has an unchecked pointer type.

Variables at external scope with unchecked pointer types can also have
bounds declared for them.   The declarations must follow the 
rules in Section~\ref{section:external-scope-variables}.

for data structures, members with unchecked pointer types can also
have bounds declared. If bounds are declared for one member of a
structure with an unchecked pointer type, they must be declared for all
members with unchecked pointer types.  Implicit conversions from \arrayptr
type to unchecked pointer type are inserted at assignments to the members.

Here are the bounds for a structure that is a counted buffer of
characters.
\begin{verbatim}
struct S {
     char *arr : count(len);
     int len;
}
\end{verbatim}

It is important to understand that the \emph{semantics of unchecked
pointers does not change in unchecked scopes even when bounds are
declared for the pointers}. The declared bounds are used  for expressions
that mix checked and unchecked pointer types. Unchecked pointer dereferences do not
have bounds checks added to them. A function that declares a bounds-safe 
interface and whose body does not use checked pointer
types is compiled as though the bounds-safe interface has been
stripped from its source code.

We next describe type annotations.   A type annotation
describes an alternate checked type to use in checked code
in place of an unchecked type.  It is used when a variable with
an unchecked pointer type should be treated as having
a \texttt{ptr} type, for example.  Syntactically, it replaces
the bounds expression in a bounds declaration.   It too
can be used at function declarations, function types,
declarations of variables with external scope, and declarations of members
of structures and unions.

A type annotation is specified using the keyword \texttt{itype} followed by 
a parenthesized type name.   The keyword \texttt{itype} is short for bounds-safe 
interface type.

We may have a method that takes a pointer to a count buffer of characters
\texttt{S} and counts the number of instances of a specific character. A type
annotation can declare that the unchecked pointer to \texttt{S}
should be treated as \verb+ptr<S>+ in checked code.
\begin{verbatim}
int count_char(S *str : ptr<S>, char arg);
\end{verbatim}

Here are examples of functions from the C Standard Library with type annotations:
\begin{verbatim}
void modf(double value, double *iptr : ptr<double>);
int fclose(FILE *stream : ptr<FILE>);
FILE *tmpfile(void) : ptr<FILE>;
struct tm *gmtime(const time_t *timer : ptr<const time_t>) : ptr<struct tm>;
\end{verbatim}
In the examples, type annotations for return types are placed 
after the parameter list, the same way that bounds expression for
return values are placed after the parameter list

A type annotation must be compatible with the original unchecked
pointer type when checkedness of pointers and bounds declarations
are ignored.  In addition, a type annotation must be at least as
checked as the original type. 

\subsection{Type checking}

Bounds-safe interfaces allow unchecked pointer types to be used
where checked pointer types with assignment compatible referent types are
expected and {\it vice versa}.
To handle this, implicit pointer conversions are inserted during type checking.
Section~\ref{section:implicit-conversions} covered implicit conversions from unchecked pointer types to checked pointer types.

Implicit conversions from checked pointer types to unchecked pointer types
with assignment-compatible referent types are allowed exactly at the uses of functions,
variables, or members with a  bounds-safe interface.  In this case, assignment
compatibility is applied in a reverse fashion.  The source referent type must be
assignment compatible with the destination referent type.  The conversions are done for rvalue expressions by inserting C cast operators to the desired unchecked types.
They may be done at:
\begin{itemize}
\item Function call arguments: If the function being called has a
      bounds-safe interface for unchecked pointer type arguments, a parameter
      type has an unchecked pointer type, the corresponding argument expression
      has a checked pointer type, and the argument referent type is assignment
      compatible with the parameter referent type, then the argument expression
      will be converted implicitly to the unchecked pointer type.
\item Assignments to a variable with external scope: if the variable being
     assigned to has an unchecked pointer type and a bounds-safe interface, the
     right-hand side expression has a checked pointer type, and the right-hand
     side expression referent type is assignment compatible with the referent
     type of the variable, then the right-hand side expression will be converted
     implicitly to the unchecked pointer type.
\item
   Member assignments: a similar conversion is done for member assignments.
\end{itemize}

Implicit conversions at bounds-safe interfaces are allowed from checked pointer types to
\uncheckedptrvoid.  This rule is likely to change in the future.  There is not a  design for
checking type-safety of casts yet and the design will amost certainly affect 
\uncheckedptrvoid\ casts.

Only one implicit pointer conversion is allowed for an expression at a bounds-safe
interface.  For example, an expression will not be
coerced implicitly from \ptrinst{\var{S}} to \uncheckedptrvoid\ to
\uncheckedptrinst{\var{T}}, where \var{S} and \var{T} are different types.

\subsection{Type compatibility}

C allows variables or functions to be declared with types that are missing information.
There can be other declarations of the variables or functions with types that are more complete
and fill in the missing information.  The types that are missing information are said to be
``compatible'' with the more complete versions of the types.  Multiple declarations of
a variable or function are required to have compatible types, not identical types.

This notion of compability extends to bounds-safe interfaces naturally.  We extend the
definition of compatibility of function types in Section~\ref{section:function-types} as follows:
\begin{itemize}
\item Given corresponding parameters with compatible unchecked pointer types,
one parameter may have a bounds declaration and one parameter may omit the bounds declaration.
\item Given return values with compatible unchecked pointer types, one
return value may have a bounds declaration and one return value may omit the bounds declaration.
\end{itemize}

The following are examples of declarations of functions with compatible types:
\begin{verbatim}
int f(int *);
int f(int * : count(5));

int g(int *a, int len, int *b);               // decl 1
int g(int *a : count(len), int len, int *b);  // decl 2
int g(int *a, int len, int *b : count(len));  // decl 3
int g(int *a : count(len), int len, int *b : count(len));  // decl 4
\end{verbatim}
The declarations of \texttt{g} are particularly interesting. They are all
compatible declarations.  The only version
that could be used in a checked scope is declaration 4.   For declarations 2 and 3,
where bounds information is missing for a parameter, the argument passed to \texttt{g}
for that parameter cannot have an \arrayptr\ type.  The typing rules prevent arguments
with checked type from being used with parameters of \texttt{g} that are missing
bounds information.

\subsection{Checking bounds declarations}
\label{section:checking-bounds-interfaces}

The checking rules in Chapter~\ref{chapter:checking-bounds} requires
only small changes to check code with bounds-safe interfaces.  The
checking is done after any implicit pointer conversions have been
inserted.
\begin{itemize}
\item Contexts are extended to include unchecked pointer variables with
      bounds-safe interfaces
\item Inference of bounds of expressions (Section~\ref{section:inferring-expression-bounds}) incorporates information from bounds-safe interfaces:
\begin{itemize}
\item Uses of variables (Section~\ref{section:checking-variables}): 
      if a variable with an unchecked pointer type is used and the context has
      a bounds for the variable, the bounds in the context are used as the bounds
      for the variable.
\item Function calls (Section~\ref{section:inferring-bounds-for-function-calls}): 
      the rules in this section are also applied when a function returns an unchecked
      pointer value and the return value has a bounds-safe interface.
\end{itemize}
\item The checking that an expression statement, declaration, or bundled
      block implies the declared bounds of variables includes:
\begin{itemize}
\item In checked contexts, all variables with bounds-safe interfaces.
\item In unchecked contexts, all variables with bounds-safe interfaces
      that are modified by an assignment within the statement, declaration,
      or bundled block, where the right-hand side expression
      is implicitly converted from a checked pointer type to an unchecked pointer type.
\end{itemize}
For the included variables, bounds are tracked through the statement or
bundled block.
\item The checking of function call arguments in
      Section~\ref{section:checking-function-call-arguments} 
      in function calls includes parameters with bounds-safe interfaces: 
\begin{itemize}
\item In checked contexts.
\item In unchecked contexts, when one or more argument expressions have been 
      converted implicitly to unchecked pointer types.
\end{itemize}
The parameters are included by adding their bounds-safe interfaces to the
declared bounds in the constructed \keyword{where} clauses.
\end{itemize}

The checking for assignments and function calls has some interesting 
implications in unchecked contexts.  It is possible to have a statement, 
a declaration, or a bundled block that mixes different kinds of pointers
in assignments to the same  variable.  
Some right-hand sides could have checked pointer types and others
could have unchecked pointer types.  This is a programming practice that is discouraged.
Nevertheless, the bounds declarations will be checked no matter which
kind of assignment is the last one.  This implies that if the last 
assignment has an unasfe pointer-typed right-hand side, it will still need
to have valid bounds.   It is also possible to mix argument expressions with 
checked and unchecked  pointer types in one function call.  The checking of bounds
for function arguments is done for all arguments, so this also implies that the unchecked
pointer-typed expressions will need to have valid bounds.

\subsection{Examples}

Here are some examples:
\begin{verbatim}
void copy(array_ptr<int> dest : count(len), 
          array_ptr<int> src : count(len), int len)
{
    // dest, src will be converted implicitly to void * pointers.
    // Even though this is an unchecked context, the function call
    // arguments will be checked to make sure that they meet the
    // bounds requirements of memcpy.  This is because the argument
    // expressions have checked pointer types.
    memcpy(dest, src, len * sizeof(int));
}
 
f(S s) 
{
     int len = s.len;
     array_ptr<char> sp : count(len) = s.arr;
     ...
     if (len > 0) {
        sp[0] = 'a';
     }
}
\end{verbatim}

This example will fail at compile time:
\begin{verbatim}
void bad_copy(int *dest,
              array_ptr<int> src : count(len), int len)
{
    // dest, src will be converted implicitly to void * pointers.
    // Because an argument had a checked pointer type, the function call
    // will be checked to make sure parameters meet the bounds
    // requirements.  This will fail because dest has no bounds.
    memcpy(dest, src, len * sizeof(int));
}
\end{verbatim}

In contrast, this example will compile:
\begin{verbatim}
void subtle_copy(int *dest,
                 array_ptr<int> src : count(len), int len)
{
    // This function call will not be checked for bounds
    // requirements because no arguments have checked pointer
    // types.  This shows both that the programmer has control
    // over checking by using types and why checked contexts
    // should be used with checked pointers when possible (perhaps
    // the programmer did not want to do this).
    memcpy(dest, (void *) src, len * sizeof(int));
}
\end{verbatim}

\section{Conversions between pointers and integers}
\label{section:pointer-integer-conversions}

C allows pointers to be converted to suitably large enough integer
types and then converted back to pointers.  Integer operations can
done on the converted values before they are converted back.
This is one of the features of C that leads people to 
say that C cannot possibly be made type-safe.  In this section,
we explain how casts between pointers and integers within 
expressions can be controlled and reasoned about.

\subsection{Tagged pointers}
We start with a motivating example, tagged pointers.
C programs sometimes store information in unused bits of pointers
to save on space in data structures.  Other times this is done
to acheive atomicity when reading or writing data.  A
program might store a tag in the least significant bits or most
significant bits of some pointers, knowing that those bits are always
zero otherwise.

The bounds of such a pointer are easy to describe: they are the bounds
of the original pointer. If a pointer variable or member is used in its
own bounds, the tag must be removed first. This is
done by masking out the bits that are always 0.

For example, suppose there is the following structure:

\begin{verbatim}
struct S {
   array_ptr<int> p : bounds(p, p + 4);
   int tag : 2;
}
\end{verbatim}

A programmer might know that on a machine with 32 bit pointers, the
least 2 significant bits of p are always 0, so the tag could be stored
within p. The bounds would be changed to:

\begin{verbatim}
struct S {
   array_ptr<int> p : bounds((array_ptr<int>) ((size_t) p & ~3), 
                             (array_ptr<int>) ((size_t) p & ~3) + 4) rel_align(char);
}
\end{verbatim}

There is one important caveat (for now): tags cannot be stored in null pointers. A
null pointer may have bounds that are not valid. This is fine because
the null pointer may not be used to access memory. However, tagging a
null pointer could result in a non-null pointer with bounds that are
invalid. This pointer could be used to access memory incorrectly.

\subsection{Extending bounds checking rules to integral
expressions}

The inference of bounds for expressions disregards non-narrowing cast operators:
the  information flows through the operation, with adjustments if the bounds
include \exprcurrentvalue.  This makes it straightforward to extend the
rules for inferring bounds to integral expressions:

\begin{itemize}
\item
  Given \texttt{\var{e1} \var{op} \var{e2}}, where \var{op} is an additive, multiplicative, or bitwise
  binary operator on integral types, the bounds of \var{e1} and \var{e2} can be
  computed:

  \begin{itemize}
  \item
    If \boundsdecl{\var{e1}}{\var{b}}, where \var{b} is not \boundsnone, and 
    \boundsdecl{\var{e2}}{\boundsnone}, and it can be proved that \texttt{\var{e1} != 0}, then
    \boundsinfer{\var{e1} \var{op} \var{e2}}{\var{b}}.
  \item
    If the special variable \exprcurrentvalue\ occurs in \var{b}:
    \begin{itemize}
    \item
      If \var{op} has an inverse operation \var{inverse-op}, then 
      \texttt{\exprcurrentvalue \var{inverse-op} \var{e2}}
      is substituted for \exprcurrentvalue.
    \item
      Otherwise the bounds of the expression are altered to be \boundsnone.
      bounds(none).
    \end{itemize}
  \item
    Similar rules apply for the reverse situation where \boundsdecl{\var{e1}}{\boundsnone}
    and \boundsdecl{\var{e2}}{\var{b}}, where \var{b} is not \boundsnone.
  \end{itemize}
\item
  Given \texttt{\var{e1} \var{op} \var{e2}} where \var{op} is a shift operator, \var{e1} is the integral value
  being shifted, and \var{e2} is the shift amount,

  \begin{itemize}
  \item
    If \boundsdecl{\var{e1}}{\var{b}}, where \var{b} is not \boundsnone, and \var{e2} has
    \boundsnone, and it can be proved that \texttt{\var{e1} != 0}, then \boundsinfer{\var{e1} \var{op} \var{e2}}{\var{b}}.
  \item
    If the special variable \exprcurrentvalue\ occurs in \var{b}.
    bounds of the expression are altered to be \boundsnone. Shift
    expressions are not invertible because they lose information.
  \end{itemize}
\end{itemize}

\subsection{An example}

The following example shows functions that create a tagged pointer and 
that set the tag to 1.

\begin{verbatim}
#define untagged_bounds(x) \
   bounds((array_ptr<int>) ((size_t) x & ~0x3), \
          (array_ptr<int>) ((size_t) x & ~0x3) + 1) rel_align(char)

array_ptr<int> create(void) 
where untagged_bounds(return_value)
{
   array_ptr<int> x : bounds(x, x + 1) = malloc(sizeof(int));
   dynamic_check(x == (array_ptr<int>) ((size_t) x & ~0x3));
   // follows from substituting the right-hand side in the current bounds
   where x : untagged_bounds(x); 
   return x;
}

// set tag to 1
array_ptr<int> set(array_ptr<int> x : untagged_bounds(x)) 
where untagged_bounds(return_value);
{
  if (x != NULL) {
     // ((size_t) x | 1) has the same bounds as x                   
     array_ptr<int> tmp : untagged_bounds(x) = (array_ptr<int>) ((size_t) x | 1);
     dynamic_check((size_t) tmp & ~0x3 == (size_t) x & ~0x3);
     // follows from substituting the right-hand side in the current bounds for tmp
     where tmp : untagged_bounds(tmp);
     return tmp;
  }
  return x;
}
\end{verbatim}

\section{Restricted interoperation with functions without prototypes}
C allows declarations of functions that do not specify the type of their parameters
(no-prototype function declarations).  This provides backward compatibility between
ANSI C from 1989 and earlier versions of C that did not check the types of
arguments at calls.  Using functions declared this way is
dangerous.  Arguments are passed based on their types and an incorrect call can be made
where the types of arguments do not match the types of the parameters of the function
definition.  This could lead to bypassing of checking.  Checked pointers could be 
converted silently to unchecked pointers or vice versa.  Even worse, parameters could contain
corrupted values or the stack could be corrupted.

We recommend strongly that programmers do not declare functions without prototypes.  This
feature is a backward compatibility feature and is used rarely now.
The GCC and clang C compilers have warning flags that will detect the declaration of functions without 
prototypes.

For checked scopes, the declaration or use of functions with no prototypes is an error
and is not allowed.   In unchecked scopes, forbidding the use of function
declarations without prototypes would violate the design goal of providing backward compatibility.
Instead, we restrict the usage of no-prototype functions to reduce the possibility of
bounds checking being bypassed accidentally in unchecked blocks.

Informally, we want to prevent values with checked types from being passed as arguments or
returned from calls to no-prototype functions.  This requires some care to define because of
structures, unions, and function pointers. We define the set of types $E$ that are an error to use
with functions without prototypes by induction.  It includes:
\begin{enumerate}
\item Checked pointer and array types.
\item Complete structure and union types with members that have types in $E$.
\item Pointers to function types that have argument or return types that are in $E$.
\item Complete structure and union types with members with bounds declarations (these
      are described in Chapter~\ref{chapter:structure-bounds}).
\item Pointers to function types with bounds declarations.
\end{enumerate}
Clauses 4 and 5 handle the case of integer-typed values that have bounds declarations
(note that bounds-safe interfaces on unchecked pointer types are not bounds declarations).

We define the following rules for unchecked scopes:
\begin{enumerate}
\item It is an error to call a function that is
declared to have no prototype and pass or return a value whose type is in $E$.
\item A function declaration with no prototype is incompatible with a function declaration
with a prototype that has parameter types or a return type in $E$ or that has bounds
declarations.
\end{enumerate}

\subsection{Examples}
The rules catch common errors but are not foolproof.  They catch passing a checked pointer
to a function with no prototype:
\begin{verbatim}
int f();

int g(ptr<int> a) {
  f(a);  // error - passing a checked type to a function without a prototype
}
\end{verbatim}
They also catch redeclaring a function with no prototype to have a checked parameter:
\begin{verbatim}
int f();

struct S {
  array_ptr<int> ap : count(len);
  int len;
}

// Error - incompatible definition of f with a prototype.
int f(S y) {
 ...
 }
\end{verbatim}
By rule 2, the definition of \verb+f+ is incompatible with the initial
declaration of \verb+f+, so this is an error.  It is an error even
if \verb+S+ is an incomplete type at the time of
a prototype declaration for \verb+f+:
\begin{verbatim} 
int f();

struct S;

int f(S y);  // Declarations involving incomplete types are allowed.

// Now define struct S.
struct S {
  array_ptr<int> ap : count(len);
  int len;
}

// Error - incompatible definition of f with the initial declaration of f.
int f(S y) {
 ...
}
\end{verbatim}

\subsection{Checking during compilation and linking}
Checking can be bypassed by code that declares a function with no prototype in one 
compilation unit and defines it in another compilation unit:
\begin{verbatim}
Compilation unit 1:

int g(ptr<int> x) {
 ...
 }

Compilation unit 2:

extern int g();
void h(void) {
  g(5);  // Error
}
\end{verbatim}
The definition of \verb+g+ in compilation unit 1 is incompatible with
the declaration in compilation unit 2, but there is no way for a compiler
to detect this.

The checking could be deferred to linking. The compiler could decorate the linker names of
functions whose argument types or return types are in $E$ differently from the names 
of functions whose argument types and return types are not in $E$.

\subsection{Unchecked pointers to checked types}
There is a limited way in which no-prototype functions can interoperate with checked types.
The definition of $E$ allows unchecked
pointers to checked pointers and arrays
to be passed to or returned from no-prototype functions.  It also allows unchecked pointers to
structures or unions that have checked members to be passed to or returned from
no-prototype functions.  Finally, it allows unchecked pointers with bounds-safe interfaces to be
passed to functions with no prototypes.  This is necessary so that bounds-safe interfaces
can be added to existing code without breaking the code.   

Here are some examples:
\begin{verbatim}
int g();

int g(ptr<int> *x);

int f();

struct S {
  array_ptr<ptr> ap : count(len);
  int len;
}

int f(S *arg);
\end{verbatim}

There are three reasons to allow unchecked pointers that point to checked data to
be passed to functions without prototypes.  First, the unchecked pointer types 
indicate a lack of checking, so it is already clear from the types of the variables being
used that there is some lack of checking.  Second, we believe that this
will support incremental conversion of code to use the Checked C extension.  Finally,
it would be difficult to enforce that an unchecked pointer does not point
to a checked type. An unchecked pointer could point to an incomplete
structure or union type. A compilation unit might never define the type. The type
could even be unresolved during linking of a library if none of the library compilation units
define the type.
