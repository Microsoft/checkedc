% !Tex root = checkedc.tex

\newcommand{\dynamicboundscast}{\texttt{dynamic\_bounds\_cast}}
\newcommand{\dynamicboundscastinst}[2]{\texttt{dynamic\_bounds\_cast<#1>#2}}
\newcommand{\assumeboundscast}{\texttt{assume\_bounds\_cast}}
\newcommand{\assumeboundscastinst}[2]{\texttt{assume\_bounds\_cast<#1>#2}}

\chapter{Interoperation}
\label{chapter:interoperation}

Code that uses checked pointer types must be able to interoperate with
code that uses unchecked pointer types.  This chapter describes support
for this.  Section~\ref{section:pointer-casting} starts with
conversion operations: how different kinds of pointers can
be converted to other kinds of pointers.  
Section~\ref{section:function-bounds-safe-interfaces}
describes how existing code that uses unchecked pointers can be modified
to present a safe interface.  The key insight is that
the interface must be both checked and unchecked, depending on context.
For existing code that uses unchecked pointers, the interface is descriptive,
but correctness is not enforced by the language.  For code that uses checked
pointers, proper usage of the interface is checked and enforced.

\section{Conversions between pointers to objects of different types}
\label{section:pointer-casting}

Conversions from a pointer to one type to a pointer to a different type
introduce two issues. First, there is type safety. Given a pointer to S
that has been converted to be a pointer to T, is it valid to treat the
memory pointed to by the pointer as being an object of type T instead of
type S? Second, there is bounds safety. Given the pointer, what range of
memory can be accessed validly using that pointer? This section focuses on
bounds safety.

Type safety is not addressed by this technical report. Of course, violating
type safety can lead to violations of bounds safety. This can happen
when there is a conversion between a checked pointer to an object of
structure type that contains a bounds-checked member and a pointer
to an object of another type. A programmer can use the pointer to the
other type to modify the bounds-checked member or its bounds in an
inconsistent fashion. For now, it is the programmer's responsibility to
update bounds-checked members and their bounds properly when using a
checked pointer that results from such a conversion. Conversions between
checked pointers to integral types, floating-point types, or structures that
contain only integral types or floating-point types cannot lead to
violations of bounds safety by themselves.

\subsection{Cast operators}
This section describes casts to pointer types.
A conversion of a value from one checked pointer type to a value of another checked
pointer type is valid when the following rules hold:
\begin{itemize}
\item If the destination pointer is a \ptrT\ type, the range of 
memory that can be accessed beginning at the source pointer is
large enough to hold a value of type \var{T}.
\item If the destination pointer is an \arrayptrT\ or \spanptrT\ type,
the bounds for the destination pointer are equal to or within the range of
the memory accessible through the source pointer.
\end{itemize}

Casts between checked pointer types should preserve bounds safety
by default.  The cases where it does not should be the unusual cases.
For this reason, C cast operations to checked pointer types are
required to be provably correct with respect to bounds at compile-time.   It
is straightforward to check casts between checked pointers to constant-sized
objects.   The rules that are used to check bounds and lightweight
invariants are used to check casts as well.   For cases where static
checking cannot prove bounds safety of casts, we add two new operators. 
Table~\ref{table:cast-operators}
describes the operators.

\begin{table}
\begin{tabular}{p{1.5in}p{4in}}
\toprule
Operator & Description \\
\texttt{(T *)} & The C cast operator.  Additional static checking rules
ensure pointer casts to checked pointer types are correct with respect to bounds.\\
\dynamicboundscast\ & Does dynamic checks to ensure bounds
safety.  It produces a runtime error if the checks fail.\\
\assumeboundscast\ & Declares new bounds for the value that are trusted without verification.  
Because it is unchecked, it is not allowed in \keyword{checked} blocks.\\
\bottomrule
\end{tabular}
\caption{Cast operators for pointer types}
\label{table:cast-operators}
\end{table}

The syntax of the new operators is similar to the syntax of C++ type
conversion operators, where the destination pointer type is specified by
placing a type argument \texttt{<\var{T}>} after the operator: \dynamicboundscastinst{\var{T}}{}
and \assumeboundscastinst{\var{T}}{}.   A relative alignment type or constant can be
specified using an optional second argument:  
\dynamicboundscastinst{\var{T}, \var{A}}{} or \assumeboundscastinst{\var{T, \var{A}}}{}.
Of course, macro-like syntax could be used as well.

For C cast operations from checked pointer types to unchecked pointer types,
it is not possible to prove that the uses of the resulting unchecked pointers
are correct at compile-time.   At the same
time, it is desirable to  catch the programming error of converting an out-of-range
checked pointer to an unchecked pointer and then trying to use the unchecked
pointer to access memory.  For this reason, we take the stance that if the
destination pointer is a \uncheckedptrT\ type, the range of memory that can be accessed
beginning at the checked pointer should be provably large enough at compile time
to hold a value of type \var{T}.

C does allow an unchecked pointer to point one element past the end of an array.
However, it is illegal to dereference that pointer.  This case mainly arises
from loops that stride through an array and end up creating a pointer one past
an element of an array.  It would be unusual for a programmer to
convert a checked pointer to an unchecked pointer and then stride downwards through an array,
predecrementing the pointer.  Because of that, C cast operations from checked pointer
types to unchecked pointer types that would create a pointer one element past
the end of the array pointed to by the checked pointer are rejected at compile time.
A programmer may use an \assumeboundscast\ operation do this instead.

\subsection{Description of cast operators}

This section describes each of the cast operators and the static
and runtime checking that is done for each operator.   

For C cast operators, the checking is only static.
Here are the rules for cast operators of the form \texttt{(\var{D}) \var{e}},
where \var{e} has source type \var{S}.  The rules are applied in addition
to any existing C typing rules that apply to the cast.

If \var{D} is not \ptrvoid\ or \uncheckedptrvoid\ and \var{D} is
\begin{itemize}
\item \ptrT: the bounds of \var{e} are computed using the rules
in Section~\ref{section:checking-nested-assignment-expressions}.
\begin{itemize}
\item If the bounds are \boundsnone, checking fails with a compile-time
error.
\item Otherwise, if the bounds are \boundsany, checking 
succeeds.
\item Otherwise, the bounds must be \bounds{\var{lb}}{\var{ub}} (or convertible
to that form).  It must be statically provable that \texttt{\var{e} >= \var{lb}}
and that {\texttt{((char *) \var{ub}) - ((char *) \var{e}) >= sizeof(\var{T})}}.    
\end{itemize}
\item \arrayptrT: No new rules are needed.   The rules in
Chapter~\ref{chapter:checking-bounds} already cover this case.
\item \spanptrT: It must be statically provable that \var{e}
and \bounds{\var{lb}}{\var{ub}} meet the relative alignment requirements of
the destination \spanptr\ type.  At runtime, a new value of
type \spanptrT\ with the value of \var{e} and \bounds{\var{lb}}{\var{ub}} will be created.
\item \texttt{\var{T} *}: If the source type \var{S} is
\begin{itemize}
\item An unchecked pointer type (\texttt{*}), checking succeeds.
\item \ptrT, the checking succeeds. This handles the case where \var{T}
is an incomplete type.
\item Otherwise, the same rules are followed as for \ptrT.
\end{itemize}
\end{itemize}

If \var{D} is \ptrvoid\ or \uncheckedptrvoid\ and \var{S} is:
\begin{itemize}
\item A \ptr\ type, static checking always succeeds.
\item An \arrayptr\ or \spanptr\ type: The bounds of \var{e} are computed:
\begin{itemize}
\item If the resulting bounds are \boundsnone, checking fails with a compile-time
error.
\item Otherwise, if the result bounds are \boundsany, checking 
succeeds.
\item Otherwise, the result must be \bounds{\var{lb}}{\var{ub}} (or convertible
to that form).   It must be statically provable that \texttt{\var{e} >= \var{lb}}
and that {\texttt{((char *) \var{ub}) - ((char *) e) >= 1}}.   
\end{itemize}
\item An unchecked pointer type:
\begin{itemize}
\item If \var{D} is \ptrvoid, the prior rule for an \arrayptr\ or \spanptr\ type is
applied.
\item If \var{D} is \uncheckedptrvoid, checking always succeeds.
\end{itemize}
\end{itemize}

The \dynamicboundscast\ operator is not strictly needed.
Programmers could write checks by hand.  However, programmers
would have to compute the bounds of expressions by hand too, so it is convenient to
have it. It takes 1 to 3 arguments, depending on the kind of conversion being done:
\begin{itemize}
\item
  \dynamicboundscastinst{\var{T}}{(\var{e1})}
  converts \var{e1} to either a \ptr\ or * type.
\item
  \dynamicboundscastinst{\var{T}}{(\var{e1},
  \var{e2})} converts \var{e1} to an \arrayptr\ or
  \spanptr\ type with bounds \boundscount{\var{e2}}.  
\item
  \dynamicboundscastinst{\var{T}}{(\var{e1},\var{e2},\var{e3})} converts \var{e1} to an
  \arrayptr\ or \spanptr\ type with bounds
  \bounds{\var{e2}}{\var{e3}}.  \dynamicboundscastinst{\var{T}, \var{A}}
  {(\var{e1},\var{e2},\var{e3})}
   optionally changes the relative alignment of the bounds to \var{A}.
\end{itemize}

Here is the checking that is done. The bounds of \var{e1} are computed
using the rules in Section~\ref{section:checking-nested-assignment-expressions}.
If the bounds of \var{e1} are \boundsnone, it is a compile-time error. 
If the bounds of \var{e1} are \boundsany, no runtime checks are needed.
Otherwise the bounds must be \bounds{\var{lb}}{\var{ub}} (or convertible to that form).
The following runtime checks are done:
\begin{itemize}
\item Check whether \var{e1} is 0.  If so, no further runtime checking
is needed.
\item Otherwise, if the operator has the form:
\begin{itemize}
\item
  \dynamicboundscastinst{\var{T}}{(\var{e1})}:
  check that there is room for least one element of \var{T} by doing
  the check for \dynamicboundscastinst{\var{T}}{(e1, 1)}.
\item
  \dynamicboundscastinst{\var{T}}{(\var{e1}, \var{e2})}:
   check that \var{lb}
  \texttt{<=} \var{e1} \texttt{\&\&} \var{e1} \texttt{+
  sizeof(referent-type(}\var{T}\texttt{)) *} e2 \texttt{<=}
  \var{ub}.
\item
  \dynamicboundscastinst{\var{T}}{(\var{e1},\var{e2},\var{e3})}: 
  check that \var{lb} \texttt{<=} \var{e2} \texttt{\&\&}
  \var{e3} \texttt{<=} \var{ub}. Also check that \var{e1}, \var{e2}, and \var{e3} 
  are relatively aligned to \var{T} (or the optional alignment parameter instead, if one
  is given). 

\end{itemize}
\end{itemize}

The operator \assumeboundscastinst{\var{T}}{} declares bounds that are trusted
without verification:
\begin{itemize}
\item
  \assumeboundscastinst{\var{T}}{(\var{e1},\var{e2})}
  converts \var{e1} to an \arrayptr\ or
  \spanptr\ type with \boundscount{\var{e2}}
\item
  \assumeboundscastinst{\var{T}}{(\var{e1},\var{e2},\var{e3})}
  converts \var{e1} to an
  \arrayptr\ or \spanptr\ type with \bounds{\var{e2}}{\var{e3}}.
  It must be statically provable that \var{e1}, \var{e2} and \var{e3}
  are relatively aligned for \var{T} (or the optional
  relative alignment parameter instead, if there is one).
\item
  \assumeboundscastinst{\var{T}}{(\var{e1})}
  converts \var{e1} to a \ptr\ or \texttt{*} type.
\end{itemize}

If any rule depends on \texttt{sizeof(\var{T})} and \var{T} is
an incomplete type, the cast operation that uses the rule shall fail
to check at compile-time.

A subtle point about the C cast operator and \dynamicboundscast\
are that they allow {\em bounds-safe casts} of an expression
\var{e} of unchecked pointer type to a \spanptr\ or
\arrayptr\ type. This is provided that the bounds for \var{e}
can be determined. Only a few kinds of expressions with unchecked pointer
types have bounds that can be determined: address-of expressions
involving variables and uses of array variables.

\subsection{Examples}
\label{section:pointer-cast-examples}

Here are examples of C-style casts to checked pointers to
constant-sized object. The static checking rules are straightforward
to apply here.  The examples assume that integers are
4 bytes in size:
\begin{verbatim}
int x = 0;
ptr<int> px = (ptr<int>) &x;
array_ptr<char> pax : count(4) = (array_ptr<char>) px;
array_ptr<char> odd_pax : count(3) = (array_ptr<char>) px;

char data[12];
ptr<int> pfirst = (ptr<int>) data; // pointed to 1st element as an integer;
array_ptr<int> pdata : count(3) = (array_ptr<int>) data;

void swizzle(ptr<int> p) {
   array_ptr<char> bytes : count(4) = (array_ptr<char>) p;
   char t0 = p[0], t1 = p[1], t2 = p[2]; t3 = p[3];
   p[0] = t3, p[1] = t2, p[2] = t1, p[3] = t0;
}
\end{verbatim}

\subsection{Implicit conversions involving checked pointers}
\label{section:implicit-conversions}

C allows implicit conversions at assignments, function call arguments,
and conditional expressions.  The purpose of
implicit conversions is to make programs shorter and easier to
read.  This section defines implicit conversions that are
allowed for checked pointer types.

The rules for checking bounds declarations are applied separately after 
typechecking to check the integrity of bounds information.  During the 
application of these rules, the implicit conversions are treated as
though they were explicit C cast operations.

\subsubsection{From unchecked pointers to checked pointers}
An expression with an unchecked pointer type can be converted implicitly to an
expression with a checked pointer type.   The destination referent type
and source referent type must be {\em assignment compatible}: they must 
either be the same type or the only difference between the two types must be that
the destination type is subject to more checking than the source type.

This can be done for the right-hand side of an assignment, a call argument,
or an arm of a conditional expression.
The type of the left-hand side of the assignment, the parameter, or the other 
arm of the conditional expression must be the checked pointer type that is the
target of the implicit conversion.

A destination referent type \var{D} is assignment compatible with a source referent
type \var{S} if
\begin{itemize}
\item \var{D} and \var{S} are compatible types.
\item \var{D} is \void\ and \var{S} is not a function type.
\item \var{D} is an array of \var{E} and \var{S} is an array of \var{T}
and:
\begin{itemize}
\item \var{D} and \var{S} are both checked arrays, both unchecked arrays, or \var{D} is a checked array and \var{S} is an unchecked array, and
\item \var{D} and \var{S} have the same number of elements, and
\item \var{E} is assignment compatible with \var{T}.
\end{itemize}
\item \var{D} is an pointer to \var{E} and \var{S} is a pointer to \var{T}
and:
\begin{itemize}
\item \var{D} and \var{S} are both \ptr\ types, both \arrayptr\ types, both unchecked pointer types,
or \var{D} is a checked pointer type and \var{S} is an unchecked pointer type
\item and \var{E} is assignment compatible with \var{T}.
\end{itemize}
\end{itemize}
{\em Discussion of assignment compatibility for function types
is deferred for now}.
C allows implicit conversions between \uncheckedptrvoid\ and other pointer
types. For now, implicit conversions from \uncheckedptrvoid\ to checked pointer
types are allowed. This may be revised later when a design
for checking type-safety of casts is complete.

\subsubsection{From checked pointers to unchecked pointers}

We must be very careful about implicit conversions from checked pointer
types to unchecked pointer types.  These kinds of conversions could allow bounds checking to
be subverted accidentally in a quiet fashion. Implicit conversions from checked pointer types to
unchecked pointer types are allowed only at bounds-safe interfaces
(Section~\ref{section:function-bounds-safe-interfaces}).

\subsubsection{From checked pointers to checked pointers}

An expression with a checked pointer type can be converted implicitly to the same kind
of checked pointer type with a \texttt{void} referent type.
This can be done for the right-hand side of an assignment, a call argument, 
and an arm of a conditional expression.  
The type of the left-hand side of the assignment, the parameter, or the other
arm of the conditional expression
must be the checked \void\ pointer type that is the target of the implicit
conversion. For example, implicit conversions from 
\ptrinst{\var{T}} to \ptrvoid\ and from \arrayptrinst{\var{T}} to \arrayptrvoid\ are allowed 
at assignments. 

Implicit conversions from checked pointers to \void\ to checked pointers to \var{T} are not allowed.
The philosophy behind this is the same one that is used in C++: places where type-safety
can be compromised by a cast should be explicit in the code.

\subsubsection{Between checked pointers and integers}

The null pointer (0) can be converted implicitly to any checked pointer type.
A checked pointer can be converted implicitly to the \texttt{\_Bool} type.

Some C compilers extend C by allowing implicit conversions between pointers
and integers or between pointers to incompatible types.  Implicit conversions
from integers to checked pointers are typically not useful in Checked C because
the checking of bounds declarations fails or the resulting pointer cannot
be used to access memory.   The rules for checking bounds declarations only
allow the target type to be \arrayptr\ type and the bounds of the expression to be
\boundsnone.

\subsubsection{Examples}

The following code illustrates examples of implicit conversions.  In these examples,
the right-hand side of an assignment has the type ``unchecked pointer to \var{T}'',
while the left-hand side has the type ``checked pointer to \var{T}''.  Implicit
conversions are allowed at all these assignments.
\begin{verbatim}
int x = 0;
ptr<int> px = &x;
array_ptr<int> pax : count(1) = &x;

int a[5] = { 0, 1, 2, 3, 4};
ptr<int> pa = a;
array_ptr<int> apa : count(5) = a; 
array_ptr<int> middle : bounds(a, a + 5) =  a + 2;
\end{verbatim}

An unchecked multidimensional array can be passed to a function expecting
a checked multi-dimensional array:
\begin{verbatim}
int b[3][3] = { { 0, 1, 2}, {3, 4, 5}, {6, 7, 8}};
extern void f(int arg checked[3][3]);
f(b);
\end{verbatim}

The mechanics behind this are somewhat complicated.  Parameters with
type ``array of \var{T}'' are actually treated as having type ``pointer to \var{T}''.
Parameters with the type ``checked array of \var{T}''
are treated as ``checked pointer to \var{T}''.  The actual type of the parameter of \texttt{f}
is
\begin{verbatim}
extern void f(array_ptr<int checked[3]> arg);
\end{verbatim}
Local variables with type ``array of \var{T}'' are treated as having 
type ``unchecked pointer to \var{T}'' when they are used.  The type of \texttt{b}
at \texttt{f(b)} is ``unchecked pointer to an array of 3 integers''.  At the call to 
\texttt{f(b)}, there is an implicit conversion from an ``unchecked pointer
to an unchecked array of 3 integers'' to a ``checked pointer to a checked array of 3
integers'':
\begin{verbatim}
int (*pb)[3] = b;
array_ptr<int checked[3]> param : count(3) = pb;
\end{verbatim}
This implicit conversion is allowed because \verb|checked int[3]| is assignment
compatible with \verb|int[3]|.
Assignment compatibility allows other interesting implicit conversions:
\begin{verbatim}
array_ptr<int[3]> t1 : bounds(b, b + 3)  = b;
array_ptr<int> t2 : bounds(b, b + 3) = t1[1];
array_ptr<int> t3 : bounds(b, b + 3) = b[1];
\end{verbatim}

Incorrect implicit conversions of conversions of pointers to constant-sized data will be rejected by
the checking of bounds declarations
\begin{verbatim}
int x = 0;
array_ptr<int> pax : count(5) : &x;     // fails to check: source not large enough
\end{verbatim}

Incorrect conversions of unchecked pointers with no bounds information
will also be rejected.  In the following example, the bounds for 
\texttt{random()} will be computed as \boundsnone, which does
not match \boundscount{1}.
\begin{verbatim}
char *random();

void f() {
    array_ptr<char> sp : count(1) = random(); 
}
\end{verbatim}

\section{Bounds-safe interfaces to existing unchecked functions and variables}
\label{section:function-bounds-safe-interfaces}

The new pointer types capture specific properties of pointers. One would
like to update existing C code to use these new pointer types. However,
this will not be possible when backward compatibility requirements
exist. Consider C runtime functions or operating system (OS) APIs. 
It may be feasible to modify the header files for runtime functions or OS APIs. 
It may be
impossible to require all uses of these functions or APIs be updated.
Code written by 3\textsuperscript{rd} parties may use these APIs and it
is not reasonable for an existing shipping OS to require all
3\textsuperscript{rd} parties update their existing code to use the new
types.

Consider what would happen if the signature for memcpy were updated to
use \arrayptr. The function

\begin{verbatim}
void *memcpy(void *dest, const void *src, size_t count);
\end{verbatim}

becomes

\begin{verbatim}
void *memcpy(array_ptr<void> dest, array_ptr<const void> src, size_t count);
\end{verbatim}

This, of course, breaks every piece of existing code that uses \texttt{memcpy}.
The code will no longer compile. C does not have method overloading, so
we cannot simply define multiple overloaded versions of \texttt{memcpy}. That
would also duplicate code and potentially increase program sizes.
The reverse problem also exists: suppose the signature for \texttt{memcpy} is not
updated. Then every ``checked'' method that calls memcpy would need to cast
the arguments to unchecked pointer types.

Given that we may not be able to change the pointer types of existing
APIs, we need to adopt an approach that supports backwards
compatibility, enables new checked code to be written easily, and maintains
the checking of new code.

We address this by:

\begin{enumerate}
\item
  Allowing programmers to declare bounds-safe interfaces to code and
  data structures that use unchecked pointers. A bounds-safe interface for
  a function, for example, describes bounds for unchecked pointer
  parameters.   Type checking is altered to insert implicit conversions
  between pointer types when needed.
\item
  In checked scopes, code is limited to using pointer types that are 
  checked pointer types or unchecked pointer types that have bounds-safe interfaces.
  This makes the code in checked scopes straightforward to understand:
  the unchecked pointer types are regarded as checked pointer types, all memory
  accesses are bounds checked or in bounds, and bounds-safe interfaces are trusted
  and respected.
\item
  In unchecked scopes, checked and unchecked pointer types can be
  be intermixed within expressions.  When that happens, the bounds-safe
  interfaces are used during checking of bounds for determining bounds
  of expressions and the required bounds for expressions. 
  Section~\ref{section:implicit-conversions} already
  explained when implicit conversions from unchecked pointer types to checked pointer
  types may be inserted.   Implicit conversions of rvalue expressions 
  from checked pointer types to unchecked pointer types are also inserted when
  there are bounds-safe interfaces that describe the bounds requirements.
\end{enumerate}

Functions that have parameters with unchecked pointer types or that return
values with unchecked pointer types can have bounds declared for the unchecked
pointer parameters or the unchecked pointer return value. Bounds must be
declared for all parameters with unchecked pointer types and the return
value, if it has an unchecked pointer type. In the case where a parameter
has \texttt{ptr} type, this can be declared specially.

Here is a bounds-safe interface for \texttt{memcpy}:
\begin{verbatim}
void *memcpy(void *dest : byte_count(len), const void *src : byte_count(len), 
             size_t len) 
             where return_value : bounds((char *) dest, (char *) dest + count)
\end{verbatim}

The correctness of bounds information is enforced at compile-time when
memcpy is passed checked pointer arguments. It is not enforced when memcpy
is passed unchecked pointer arguments.

Similarly, for data structures, members with unchecked pointer types can
have bounds declared. If bounds are declared for one member of a
structure with an unchecked pointer type, they must be declared for all
members with unchecked pointer types.

Here are the bounds for a structure that is a counted buffer of
characters.
\begin{verbatim}
struct S {
     char *arr : count(len);
     int len;
}
\end{verbatim}

We may have a method that takes a counted buffer of characters and
counts the number of instances of a specific character. The \texttt{ptr}
declaration can be used to declare an unchecked pointer to a singleton
object of a type:
\begin{verbatim}
int count_char(S *str where ptr, char arg);
\end{verbatim}

Variables at external scope with unchecked pointer types may also have
bounds declared for them.   The declarations must follow the 
rules in Section~\ref{section:external-scope-variables}.

It is important to understand that the \emph{semantics of unchecked
pointers does not change in unchecked scopes even when bounds are
declared for the pointers}. The declared bounds are used  for expressions
that mix checked and unchecked pointer types. Unchecked pointer dereferences do not
have bounds checks added to them. A function that declares a bounds-safe 
interface and whose body does not use checked pointer
types is compiled as though the bounds-safe interface has been
stripped from its source code.

\subsection{Type checking}

Bounds-safe interfaces allow unchecked pointer types to be used
where checked pointer types with compatible referent types are expected
and {\it vice versa}.
To handle this, implicit pointer conversions are inserted during type checking.
Section~\ref{section:implicit-conversions} covered implicit conversions from unchecked pointer types to checked pointer types.

Implicit conversions from checked pointer types to unchecked pointer types
with compatible referent types are allowed exactly at the uses of functions,
variables, or members with a  bounds-safe interface.  The conversions are done for rvalue expressions by inserting C cast operators to the desired unchecked types.
They may be done at:
\begin{itemize}
\item Function call arguments: If the function being called has a 
      bounds-safe interface for unchecked pointer type arguments, a parameter type
      has an unchecked pointer type, and the corresponding argument expression has a checked
      pointer type whose referent type is compatible with the parameter
      referent type, the argument expression will be converted implicitly to the 
      unchecked pointer type.
\item Assignments to a variable with external scope: if the variable being
     assigned to has an unchecked pointer type and a bound-safe interface and the
     right-hand side expression has a checked pointer whose referent type is
     compatible with the
     variable referent type, the right-hand side expression will be converted
     implicitly to the unchecked pointer type.
\item
   Member assignments: a similar conversion is done for member assignments.
\end{itemize}

Implicit conversions at bounds-safe interfaces are allowed from checked pointer types to
\uncheckedptrvoid.  This rule is likely to change in the future.  There is not a  design for
checking type-safety of casts yet and the design will amost certainly affect 
\uncheckedptrvoid\ casts.

Only one implicit pointer conversion is allowed for an expression at a bounds-safe
interface.  For example, an expression will not be
coerced implicitly from \ptrinst{\var{S}} to \uncheckedptrvoid\ to
\uncheckedptrinst{\var{T}}, where \var{S} and \var{T} are different types.


\subsection{Checking bounds declarations}
\label{section:checking-bounds-interfaces}

The checking rules in Chapter~\ref{chapter:checking-bounds} requires
only small changes to check code with bounds-safe interfaces.  The
checking is done after any implicit pointer conversions have been
inserted.
\begin{itemize}
\item Contexts are extended to include unchecked pointer variables with
      bounds-safe interfaces
\item Inference of bounds of expressions (Section~\ref{section:inferring-expression-bounds}) incorporates information from bounds-safe interfaces:
\begin{itemize}
\item Uses of variables (Section~\ref{section:checking-variables}): 
      if a variable with an unchecked pointer type is used and the context has
      a bounds for the variable, the bounds in the context are used as the bounds
      for the variable.
\item Function calls (Section~\ref{section:inferring-bounds-for-function-calls}): 
      the rules in this section are also applied when a function returns an unchecked
      pointer value and the return value has a bounds-safe interface.
\end{itemize}
\item The checking that an expression statement, declaration, or bundled
      block implies the declared bounds of variables includes:
\begin{itemize}
\item In checked contexts, all variables with bounds-safe interfaces.
\item In unchecked contexts, all variables with bounds-safe interfaces
      that are modified by an assignment within the statement, declaration,
      or bundled block, where the right-hand side expression
      is implicitly converted from a checked pointer type to an unchecked pointer type.
\end{itemize}
For the included variables, bounds are tracked through the statement or
bundled block.
\item The checking of function call arguments in
      Section~\ref{section:checking-function-call-arguments} 
      in function calls includes parameters with bounds-safe interfaces: 
\begin{itemize}
\item In checked contexts.
\item In unchecked contexts, when one or more argument expressions have been 
      converted implicitly to unchecked pointer types.
\end{itemize}
The parameters are included by adding their bounds-safe interfaces to the
declared bounds in the constructed \keyword{where} clauses.
\end{itemize}

The checking for assignments and function calls has some interesting 
implications in unchecked contexts.  It is possible to have a statement, 
a declaration, or a bundled block that mixes different kinds of pointers
in assignments to the same  variable.  
Some right-hand sides could have checked pointer types and others
could have unchecked pointer types.  This is a programming practice that is discouraged.
Nevertheless, the bounds declarations will be checked no matter which
kind of assignment is the last one.  This implies that if the last 
assignment has an unasfe pointer-typed right-hand side, it will still need
to have valid bounds.   It is also possible to mix argument expressions with 
checked and unchecked  pointer types in one function call.  The checking of bounds
for function arguments is done for all arguments, so this also implies that the unchecked
pointer-typed expressions will need to have valid bounds.

\subsection{Examples}

Here are some examples:
\begin{verbatim}
void copy(array_ptr<int> dest : count(len), 
          array_ptr<int> src : count(len), int len)
{
    // dest, src will be converted implicitly to void * pointers.
    // Even though this is an unchecked context, the function call
    // arguments will be checked to make sure that they meet the
    // bounds requirements of memcpy.  This is because the argument
    // expressions have checked pointer types.
    memcpy(dest, src, len * sizeof(int));
}
 
f(S s) 
{
     int len = s.len;
     array_ptr<char> sp : count(len) = s.arr;
     ...
     if (len > 0) {
        sp[0] = 'a';
     }
}
\end{verbatim}

This example will fail at compile time:
\begin{verbatim}
void bad_copy(int *dest,
              array_ptr<int> src : count(len), int len)
{
    // dest, src will be converted implicitly to void * pointers.
    // Because an argument had a checked pointer type, the function call
    // will be checked to make sure parameters meet the bounds
    // requirements.  This will fail because dest has no bounds.
    memcpy(dest, src, len * sizeof(int));
}
\end{verbatim}

In contrast, this example will compile:
\begin{verbatim}
void subtle_copy(int *dest,
                 array_ptr<int> src : count(len), int len)
{
    // This function call will not be checked for bounds
    // requirements because no arguments have checked pointer
    // types.  This shows both that the programmer has control
    // over checking by using types and why checked contexts
    // should be used with checked pointers when possible (perhaps
    // the programmer did not want to do this).
    memcpy(dest, (void *) src, len * sizeof(int));
}
\end{verbatim}

\section{Conversions between pointers and integers}
\label{section:pointer-integer-conversions}

C allows pointers to be converted to suitably large enough integer
types and then converted back to pointers.  Integer operations can
done on the converted values before they are converted back.
This is one of the features of C that leads people to 
say that C cannot possibly be made type-safe.  In this section,
we explain how casts between pointers and integers within 
expressions can be controlled and reasoned about.

\subsection{Tagged pointers}
We start with a motivating example, tagged pointers.
C programs sometimes store information in unused bits of pointers
to save on space in data structures.  Other times this is done
to acheive atomicity when reading or writing data.  A
program might store a tag in the least significant bits or most
significant bits of some pointers, knowing that those bits are always
zero otherwise.

The bounds of such a pointer are easy to describe: they are the bounds
of the original pointer. If a pointer variable or member is used in its
own bounds, the tag must be removed first. This is
done by masking out the bits that are always 0.

For example, suppose there is the following structure:

\begin{verbatim}
struct S {
   array_ptr<int> p : bounds(p, p + 4);
   int tag : 2;
}
\end{verbatim}

A programmer might know that on a machine with 32 bit pointers, the
least 2 significant bits of p are always 0, so the tag could be stored
within p. The bounds would be changed to:

\begin{verbatim}
struct S {
   array_ptr<int> p : bounds((array_ptr<int>) ((size_t) p & ~3), 
                             (array_ptr<int>) ((size_t) p & ~3) + 4) rel_align(char);
}
\end{verbatim}

There is one important caveat (for now): tags cannot be stored in null pointers. A
null pointer may have bounds that are not valid. This is fine because
the null pointer may not be used to access memory. However, tagging a
null pointer could result in a non-null pointer with bounds that are
invalid. This pointer could be used to access memory incorrectly.

\subsection{Extending bounds checking rules to integral
expressions}

The inference of bounds for expressions disregards non-narrowing cast operators:
the  information flows through the operation, with adjustments if the bounds
include \exprcurrentvalue.  This makes it straightforward to extend the
rules for inferring bounds to integral expressions:

\begin{itemize}
\item
  Given \texttt{\var{e1} \var{op} \var{e2}}, where \var{op} is an additive, multiplicative, or bitwise
  binary operator on integral types, the bounds of \var{e1} and \var{e2} can be
  computed:

  \begin{itemize}
  \item
    If \boundsdecl{\var{e1}}{\var{b}}, where \var{b} is not \boundsnone, and 
    \boundsdecl{\var{e2}}{\boundsnone}, and it can be proved that \texttt{\var{e1} != 0}, then
    \boundsinfer{\var{e1} \var{op} \var{e2}}{\var{b}}.
  \item
    If the special variable \exprcurrentvalue\ occurs in \var{b}:
    \begin{itemize}
    \item
      If \var{op} has an inverse operation \var{inverse-op}, then 
      \texttt{\exprcurrentvalue \var{inverse-op} \var{e2}}
      is substituted for \exprcurrentvalue.
    \item
      Otherwise the bounds of the expression are altered to be \boundsnone.
      bounds(none).
    \end{itemize}
  \item
    Similar rules apply for the reverse situation where \boundsdecl{\var{e1}}{\boundsnone}
    and \boundsdecl{\var{e2}}{\var{b}}, where \var{b} is not \boundsnone.
  \end{itemize}
\item
  Given \texttt{\var{e1} \var{op} \var{e2}} where \var{op} is a shift operator, \var{e1} is the integral value
  being shifted, and \var{e2} is the shift amount,

  \begin{itemize}
  \item
    If \boundsdecl{\var{e1}}{\var{b}}, where \var{b} is not \boundsnone, and \var{e2} has
    \boundsnone, and it can be proved that \texttt{\var{e1} != 0}, then \boundsinfer{\var{e1} \var{op} \var{e2}}{\var{b}}.
  \item
    If the special variable \exprcurrentvalue\ occurs in \var{b}.
    bounds of the expression are altered to be \boundsnone. Shift
    expressions are not invertible because they lose information.
  \end{itemize}
\end{itemize}

\subsection{An example}

The following example shows functions that create a tagged pointer and 
that set the tag to 1.

\begin{verbatim}
#define untagged_bounds(x) \
   bounds((array_ptr<int>) ((size_t) x & ~0x3), \
          (array_ptr<int>) ((size_t) x & ~0x3) + 1) rel_align(char)

array_ptr<int> create() 
where untagged_bounds(return_value)
{
   array_ptr<int> x : bounds(x, x + 1) = malloc(sizeof(int));
   dynamic_check(x == (array_ptr<int>) ((size_t) x & ~0x3));
   // follows from substituting the right-hand side in the current bounds
   where x : untagged_bounds(x); 
   return x;
}

// set tag to 1
array_ptr<int> set(array_ptr<int> x : untagged_bounds(x)) 
where untagged_bounds(return_value);
{
  if (x != NULL) {
     // ((size_t) x | 1) has the same bounds as x                   
     array_ptr<int> tmp : untagged_bounds(x) = (array_ptr<int>) ((size_t) x | 1);
     dynamic_check((size_t) tmp & ~0x3 == (size_t) x & ~0x3);
     // follows from substituting the right-hand side in the current bounds for tmp
     where tmp : untagged_bounds(tmp);
     return tmp;
  }
  return x;
}
\end{verbatim}