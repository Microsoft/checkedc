% !Tex root = checkedc.tex

\parbox{5.5in}{
System programmers need a system programming language that
detects or prevents common programming errors involving pointers. This
will improve the reliability and security of system software and
the productivity of system programmers. This technical report describes
Checked C, an extended version of C that provides a way to write C code that
is guaranteed to be bounds-checked and less prone to type confusion.
}

\vspace{11pt}

\parbox{5.5in}{
Checked C adds new pointer types and array types that are
bounds-checked, yet layout-compatible with existing pointer and array
types. In keeping with the low-level nature of C, 
programmers control the placement of bounds information in data
structures and the flow of bounds information through programs. Static
checking enforces the integrity of the bounds information and allows the
eliding of some dynamic checking. Dynamic checking enforces the
integrity of memory accesses at runtime when static checking cannot.
}

\vspace{11pt}

\parbox{5.5in}{
Checked C also extends the C type system so that it provides
improved type safety, focusing on reducing type
confusion due to void pointers.  It adds opaque types, generic
functions and structures, and hidden types.  This lets
most void pointer uses be replaced with type-safe code.
The use of generic functions and structures is constrained to avoid 
the need for code cloning.
}

\vspace{11pt}

\parbox{5.5in}{
Checked C is backwards-compatible: existing C programs work
``as is''. Programmers incrementally opt-in to bounds checking, while
maintaining binary compatibility.
Checked C introduces the notion of checked scopes to ensure
that regions of code do not use unchecked pointers or type casts
involving void pointers.
}


