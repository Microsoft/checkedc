% !Tex root = checkedc.tex

\chapter{Introduction}
\label{chapter:introduction}

The C programming language \cite{Ritchie1988, ISO2011} allows programmers to use 
pointers directly. A pointer is an address of a location in memory. Programs may do
arithmetic on pointers, dereference them to read memory, or assign
through them to modify memory. The ability to use pointers directly
makes C well-suited for low-level system programming that is ``close to
the hardware'' and allows programmers to write efficient programs. C
also unifies pointer types and array types. They can usually be used
interchangeably and array subscripting is a synonym for equivalent
pointer operations.

Pointers and the unification of arrays and pointers are one of the
strengths of the C programming language, allowing programmers to write
concise, efficient programs. At the same time, they are a source of
reliability and security problems in modern software. This is
because pointers and array indices are not bounds checked in C and
related languages such as C++. Bounds checking checks that a pointer or
array index is in bounds before it is used to read or write memory. A
pointer to an array object is in bounds if it points to an element of
the array object. An array index is in bounds if the index is greater
than or equal to zero and less than the size of the array.

Between 2010 and 2015, buffer overflows accounted for between 10-16\% of
publicly reported security vulnerabilities in the U.S. National
Vulnerability Database each year \cite{NIST2015}. The vulnerabilities have affected
software implemented in C and C++ that is widely used, including the
Windows and Linux operating systems, the Internet Explorer, Chrome, and
Safari web browsers, the Apache web server, the OpenSSL security
library, scripting language implementations for Bash, Ruby, and PHP, and
media playback software such as QuickTime.

Because pointers and array indices are not bounds checked in C, a
programming error involving them may corrupt memory locations used by
the program. The memory locations may hold data that is important to the
computations being done by the program or data that is essential to the
control-flow of the program, such as return address locations and
function pointers. Memory corruption can lead to a program producing
incorrect results or, in the hands of a malicious adversary, the
complete malfunctioning of the program and the takeover of a running
process by the adversary.

This technical report describes Checked C, an extension to C that provides
bounds checking for pointers and arrays. There are two obstacles to
adding bounds checking to C. First, it is not clear where to put the
bounds information at runtime. Second, it is not clear how to make the
bounds checking efficient for programs where performance matters. The
solution of changing the representation of all C pointer types and
arrays to carry bounds information is not sufficient in practice. C may
be used at the base of systems where hardware or standards dictate data
layout and data layout cannot be changed. C programs must also
interoperate with existing operating systems and software that require
specific data layouts.

\section{Overview}
Checked C addresses the bounds checking problem for C by:

\begin{itemize}
\item
  Introducing different pointer types to capture the different ways in
  which pointers are used. The unchecked C pointer type \texttt{*} is kept
  and three new {\it checked} pointer types are added: one for pointers
  that are never used in pointer arithmetic and do not need bounds
  checking (\ptr),  one for \emph{array pointer types} that are involved
  in pointer arithmetic and need bounds checking (\arrayptr), and one
  for \emph{array pointer types where the array is null-terminated}
  (\ntarrayptr).
\item
  For array pointer types (\arrayptr), in keeping with the low-level
  nature of C,  bounds checking is placed under programmer control. 
  This differs from languages like
  Java, where bounds checking is completely automatic. A programmer
  declares \emph{bounds}, where the bounds for an \arrayptr\
  variable are given by non-modifying C expressions. These are a subset
  of C expressions that do not modify variables or memory. They include
  local variables, parameters, constant expressions, casts, and
  operators such as addition, subtraction, and address-of (\texttt{\&})
  operators. Static checking ensures that programs declare and maintain
  bounds information properly. The bounds are used at runtime to enforce
  bounds safety, if necessary.
\item
  Null-terminated array types (\ntarrayptr) extend 
  array pointer types.  An \ntarrayptr\ variable points to
  an array that is followed in memory by a sequence of elements
  that ends with a null terminator.  The runtime bounds checking
  and the static checking are extended to handle the sequence.
  This allows C string-processing code to be bounds checked.
\item
  Introducing different array types to distinguish between arrays whose
  accesses are bounds-checked and existing C arrays whose accesses are
  not bounds-checked. A programmer places the modifier \keyword{checked}
  before the declaration of the bound(s) of the array: \texttt{int x
  checked[5][5]} declares a 2-dimensional array for which all
  accesses will be bounds-checked.  Arrays with null terminators
  as their last element can be declared using \keyword{nt\_checked}
  instead of \keyword{checked}.
\item
  For structure types with \arrayptr -typed members, a
  programmer declares \emph{member bounds} for those members. A member
  bounds declares the bounds for a member in terms of members of the
  structure type. Member bounds can be suspended temporarily for
  specific variables and objects. Static checking ensures that updates
  to members maintain the member bounds.   A programmer may declare
  bounds for \ntarrayptr -typed embers also, although that is often
  not necessary.
\item
  Introducing bounds-safe interfaces to address the problem of
  interoperation between checked code and unchecked code. A bounds-safe
  interface describes the checked interface to unchecked code by declaring
  bounds for unchecked pointers in function signatures and data structures.
  It describes a boundary that is ``checked'' or ``unchecked''
  depending on what kind of code is using it. The
  interface is trusted in checked code (code that uses only checked pointer types).
  Proper usage is enforced via checking at compile time and runtime. For
  code that uses only unchecked pointer types, the interface is descriptive and
  not enforced by
  language checking. This provides a way to upgrade existing code to
  provide a checked interface without breaking existing users of the code.
\item
  Introducing checked program scopes, where bounds checking is the
  default behavior. In a checked program scope, definitions of variables
  and functions can use checked pointer types and cannot use unchecked pointer
  types. Declarations involving unchecked pointer types must provide
  bounds-safe interfaces. Checked program scopes avoid problems with
  subtle misuse of bounds-safe interfaces.
\item
  Reasoning about the correctness of programs with declared bounds
  sometimes requires reasoning about simple aspects of program behavior.
  To support this, lightweight invariants are added to Checked C. A lightweight
  invariant declares a relation between a variable and a simple
  expression using a relational operator. An example would be the
  statement \texttt{x < y + 5}. Lightweight invariants can be
  declared at variable declarations, at assignment statements, for
  parameters, and for return values. Checked C is extended with rules for
  checking these lightweight invariants. Just as type checking rules are
  defined by the programming language, so are rules for checking
  lightweight invariants. The checking of the correctness of
  programmer-declared bounds is integrated with the checking of
  invariants.
\item
  For the cases where static checking reaches it limits, a programmer
  can introduce dynamic checks that are runtime errors if they fail.
  Dynamic checks use the syntax \keyword{dynamic\_check} \var{e}, where \var{e} is an
  integer-valued expression. A check is similar to an assert, except
  that it is never removed from the program (unless a compiler proves
  it is redundant). It cannot be removed because the integrity of the
  program depends upon it.
\end{itemize}

For an existing C program to be correct, there has to be an
understanding on the part of the programmer as to why pointers and array
indices stay in range. The goals of the design are to let the programmer
write this knowledge down, to formalize existing practices, such as
array pointer parameters being paired with length parameters, and to
check this information.

To simplify bounds and reasoning about bounds for
\arrayptr\ types, pointer arithmetic overflow for \arrayptr\
types is considered a runtime error. Pointer arithmetic involving a null
pointer for \arrayptr\ or \ntarrayptr types is also a runtime error.

Efficiency is addressed by extending the static checking so that it can
guarantee that specific bounds checks will always succeed at runtime for
\arrayptr\ . The static checking
supports the scenario of simple control-flow enclosing the bounds check
guaranteeing that the bounds check will succeed. For example, a for-loop
may iterate only over values within the declared bounds of an
\arrayptr\ variable.

A problem with incorporating static checking into a programming language
is that static checking needs to be something that compilers can do
quickly and deterministically. Static checking can become very expensive
to do, depending on the language of invariants and the inference that
compilers are expected to do. For example, Presburger arithmetic is
integer arithmetic restricted only to addition and less than or equal
operations. It is NP-complete to determine whether a formula in the
first-order logic for quantifier-free Presburger arithmetic is
satisfiable (true or false). Even statically checking properties of
simple fragments of real programs can be computationally intractable.

This problem is addressed in two ways. First, the language rules that are
used to check the validity of bounds are limited intentionally.  The rules can
check the validity of bounds that are needed in practice.  However, there are
bounds that are true that cannot be proved using the rules.  In the terminology
of program logics, the rules are incomplete.  As an example, the 
rules about distributivity limit the size of
expressions that they produce.   In practice, this means that 
simple disjunctive bounds can be checked, but complex disjunctive bounds cannot be checked.
 
Second, the inference that compilers do for checking program invariants is
limited.  Compilers act as \emph{checkers} for invariants. They check that
declared invariants follow from other declared invariants, the
program control-flow, intervening assignments, and simple axioms about
invariants, such as transitivity of relational operators. Compilers
do not try to devise invariants or prove the correctness of
invariants; they apply simple local reasoning to check them. The
programmer has to call out the relevant facts. If a programmer declares
an invariant \texttt{x == y} but neglects to declare the invariant
\texttt{y == z}, a checker may not be able to reason several
statements later that \texttt{x == z}, even though it may be true at
that point in the program. This is taking advantage of the fact that
checking a proof is usually much easier than creating the proof.

Establishing the bounds-safety of pointer operations is just the first step in
establishing type safety for C programs. There are other ways which C
programs may fail. C programs may incorrectly deallocate memory that is
still in use, do incorrect type-unsafe pointer casts, or have concurrency
races that tear data structures. Addressing these problems is beyond the
scope of this technical report. For now, it is assumed that programs are
correct in these other aspects.

This design is being done in an iterative fashion.  To validate the
design, we mocked up modifying a subset of the OpenSSL 
code base \cite{OpenSSL2015} to be bounds-safe .  
We created C++ templates for the new pointer types and modified OpenSSL to 
compile as valid C++ code.
We hand-edited about 11,000 lines of the code to use checked pointer
types with full bounds annotations.  We used macros to encapsulate 
the bounds annotations so that they could be elided from the code
and OpenSSL compiled and tested using the new types.  We also modified the
generic stack type in OpenSSL to use the \texttt{ptr} type, which required
cross-cutting changes across the code base (in all, about 160 files were changed) 
as well dealing with complicated macros.   

We learned the following from this
experience.   First, it was important to have a compact, succinct syntax
for declaring bounds. 
Second, in most cases, declaring bounds at declarations was sufficient for 
tracking the bounds of variables.  Large blocks of code 
remained unchanged, which matches the observations of the
Cyclone project \cite{Jim2002}, an earlier research 
effort to create a type-safe version of C.
Third, the expressions allowed in declaring
bounds needed to be rich.  Fourth, pointer casts were used
fairly extensively, but often times it was obvious that the casts were
correct with respect to bounds.  The existing bounds could be modified
easily to be appropriate for the new referent type of the pointer.
Fifth, it was clear that there needed
to be a graceful way of interoperating with existing libraries that could
not be changed.   Finally, signed integer overflow was a pervasive
possibility, which raised questions about the meaning of bounds 
declarations that used signed integers. 

We revised the design to
address these issues.  In particular, we paid close attention to
tracking bounds through pointer casts.  We also made sure that the
contraints on signed integer expressions used in bounds expressions
were understood and could be written down in the language of
simple invariants.

The design is a work-in-progress.  Some material will be missing
or incomplete because the language design has not yet been done.
Readers should be aware that all parts of the document are subject to change. 
We are interested in feedback and suggestions about ways to improve
the design.

\section{Principles for extensions}
\label{chapter:principles}

Here are the principles that are followed to extend C to support bounds checking:

\begin{enumerate}
\item
  Preserve the efficiency and control of C. C is designed to be
  low-level and work with the same types that computer processors work
  with. This allows programmers to
  control what programs do precisely at the
  machine level. This efficiency and control are reasons why C is valued as a
  system programming language. Extensions will be ``pay-as-you-go'' and
  continue to provide precise control to programmers at the machine
  level. Hidden costs will be avoided.
\item
  Be Minimal. This means adding the minimal set of extensions needed to
  accomplish the goals. It is easier to learn extensions if there are as
  few of them as possible. It also stays true to the design goals
  of C.
\item
  Aim for clarity and succinctness. Clarity means that code is easy to
  understand and extensions are straightforward to understand.
  Succinctness means the programmers have less to read or type.
  Programmers value clarity and succinctness because it makes them more
  productive.  Sometimes clarity and succinctness are in
  tension and sometimes they are not. When they are in conflict, clarity
  will be prioritized above succinctness, primarily because source code
  is usually read many more times than it is written.
\item
  Enable incremental use. Real systems are large and complicated, with
  hundreds of thousands and millions of lines of code. The teams that
  work on those systems will adopt checked pointer operations over time,
  not all at once, so incremental use of checked pointer operations will be
  supported. Teams will prefer incremental conversion paths because of
  practical matters such as reducing risk, fixing existing bugs
  identified by introducing bounds checking, maintaining system
  stability, and understanding performance effects. Even though
  incremental use will be supported, it is not the end goal. We believe
  that benefits of using checked pointer operations will be modest until
  almost an entire system is converted. At that point, we expect a
  qualitative increase in system reliability and programmer
  productivity.
\end{enumerate}

Two specific design principles are adopted based on these principles:

\begin{enumerate}
\item
  Do not change the meaning of existing C code. Methods that do not use
  extensions will continue to compile, link, and run ``as is''. If the
  meaning of existing C code is changed, it will violate the principles
  of clarity and enabling incremental adoption.
\item
  Adopt existing notations from C++ when it meets our needs, instead of
  inventing new notations. Many systems are hybrid C/C++ systems, so
  this approach fits with the principle of clarity. It also enables
  incremental adoption. One of the design goals of C++ has been that C
  is a subset of C++.  It is a design goal to allow Checked C to be a 
  subset of C++ too.
\end{enumerate}

\section{Organization of this document}

The target audience for this document includes programmers interested in 
learning about Checked C, compiler writers, and language designers.
These groups have different and conflicting needs.  Programmers will find that
some sections have too much detail, while language designers will likely
find that some sections do not have enough detail.  Language designers
may be less interested in compiler implementation details.
We first describe the organization of the document and then provide
a roadmap for the different audiences.

Chapter~\ref{chapter:core-extensions} describes
the new pointer types and the new array types, including syntax,
semantics, and error conditions. It also covers other extensions to C
semantics. One extension is the introduction of checked program scopes to
prevent inadvertent use of unchecked types. Another extension is a
refinement of C's notion of undefined behavior. We wish to maintain
bounds safety even in the face of integer overflow, which is a
widespread possibility in C programs. This requires a more nuanced
requirement on the result of overflow than ``the result is undefined.''

Chapters~\ref{chapter:core-extensions} through 
\ref{chapter:pointers-to-data-with-arrayptrs}
present the extensions to C for declaring bounds
for \arrayptr\ values stored in variables and structures and
checking the consistency of those declared bounds. The extensions are
organized by C language feature, moving from simple language features to
more complicated language features. This allows a reader to understand
concepts one at time.

Chapter~\ref{chapter:tracking-bounds} describes how programmers declare the bounds for
\arrayptr\ variables and the meaning of bounds declarations.
Bounds can be declared at variable declarations. They can also be
declared for automatic (local) variables at assignments. If the only
bounds declaration for a variable is at the declaration of the variable,
the bounds declaration is an invariant bounds declaration. An invariant
bounds declaration must be true for the lifetime of the variable. If
there are bounds declarations for the variable at assignments or other
declarations, all bounds declarations for the variable are
dataflow-sensitive. Dataflow-sensitive bounds declarations extend via
dataflow to uses of the variable.

Invariant bounds declarations must usually be valid after every
statement and declaration in the scope of a variable. Sometime multiple
statements are needed to update the set of a variables used in an
invariant bounds declaration. To support this, expression statements and
declarations can be \emph{bundled}, in which case bounds declarations
must be valid only at the end of the bundle. Within the bundled block,
the variables may be inconsistent with respect to bounds declarations
that use them. The variable can be updated so that consistency is
re-established at the end of the bundled block.

Chapter~\ref{chapter:checking-bounds} describes rules that the compiler uses to check the validity
of bounds declarations. It covers inferring bounds for expressions.
Because expression may have assignments embedded within them, it also
covers inferring effects of an expression on the bounds of variables.
This inferred information is then used to validate that declarations and
statements correctly declare bounds and maintain the bounds information.

Chapter~\ref{chapter:interoperation} covers interoperation between 
checked and unchecked code. It covers conversions between checked and
unchecked pointers, as well as conversions between the new kinds of checked pointers.
It pins down the notion of checked code  and unchecked code. Finally, it covers
bounds-safe interfaces in depth.

Chapter~\ref{chapter:structure-bounds} extends these ideas from variables to data 
structures by
introducing member bounds, which are type-level invariants about members
of structure types. For now, it assumes that the programmer has done
concurrency control around shared data properly and ignores the fact
that data structures may be modified in racy ways such that invariants
are violated.  The chapter also describes the rules that the compiler
uses to check programs with member bounds.

Chapter~\ref{chapter:pointers-to-data-with-arrayptrs} describes how
to extend reasoning about bounds to pointers to data that has \arrayptr-typed
values within it.  This includes pointers to structures with members that
have \arrayptr\ types and pointers directly to \arrayptr s.  Pointers to 
structures can be used easily by ensuring that modifications to members
preserve type-level bounds invariants. 
This follows the lead of the Deputy system \cite{Condit2007}. 
Complexity arises for pointers to \arrayptr s where 
the \arrayptr s have bounds that depend on other variables or members.  The 
other variables or members need to be abstracted by pointers as well to ensure 
that invariants for bounds are not violated.  Assignments via pointer
indirections need to be coordinated to maintain the invariant.  This
gives rise to constraints on the pointers and the variables or
members whose addresses have been taken.

Chapter~\ref{chapter:simple-invariants}
extends the checking of bounds to incorporate simple
reasoning about bounds and program behavior. It includes a set of rules
for deducing facts that are true about a program at a specific point in
the program (for example, given an assignment \texttt{x = y;} the fact
that x == y is true after the assignment). Facts can also be deduced
from program control-flow. There are additional rules for reasoning
about whether one fact is true given a set of other facts (for example,
given x == y and the statement \texttt{z = y;} z == x is true after that
statement). These rules and facts can be used to deduce the correctness
of bounds declarations that differ from those inferred directly by the
checking described in Chapters~\ref{chapter:checking-bounds} 
and~\ref{chapter:pointers-to-data-with-arrayptrs}. For example, 
a programmer may wish to narrow the memory that is accessible via an
\arrayptr\ variable by declaring bounds that are a subrange of
the bounds inferred by the checking. A programmer may wish to update the
bounds for an \arrayptr\ variable after an assignment \texttt{x
= y}, substituting x for y.

The same static checking that is used for bounds can be used to reason
about the ranges of variables at specific points in a program. From
there, it is a short step to deducing at compile-time that bounds are
always satisfied at a particular memory access in a program. For
example, a fact can be that the range of an integer variable \texttt{i}
is always between 0 and 10. This can be used to deduce that an array
access in a crucial inner loop is always in bounds.

Chapter~\ref{chapter:related-work} describes related work that addresses
the lack of bounds checking in C.  Because of the serious practical consequences
for computer security and software reliability, there has been extensive work
in the area.  We are heavily influenced by the Deputy system \cite{Feng2006,Condit2007}.
\omitted{
Chapter~\ref{chapter:eval} evaluates the design by describing our experience modifying
an existing C open-source code base by hand to use the Checked C
extensions. We chose to modify OpenSSL, an important widely-used
open-source code base. The static and dynamic checking is not
implemented in a compiler yet, so we cannot be sure of the correctness
of the modifications or understand the practical benefits of checking.
Still, this gives some idea about the usefulness and applicability of
the design.

Chapter~\ref{chapter:open-issues} summarizes the open problems uncovered by 
Chapter~\ref{chapter:eval} or
unaddressed by the design. It also describes next steps for implementing
this in a C compiler.
}
Chapter~\ref{chapter:open-issues} summarizes open issues that remain to be addressed
by the design.
Chapter~\ref{chapter:design-alternatives} discusses design alternatives
that were considered and not chosen.  It explains why those alternatives were
not chosen.

Here are the suggested roadmaps for the different groups of readers.
For programmers interested in learning about Checked C, 
we suggest reading Chapter~\ref{chapter:core-extensions} except 
Sections~\ref{section:pointer-arithmetic-errors} and \ref{section:changes-to-undefined-behavior},
Chapter~\ref{chapter:tracking-bounds}
except Section~\ref{section:extent-of-declarations}, 
Section~\ref{section:inferring-expression-bounds} of Chapter~\ref{chapter:checking-bounds},
Chapter~\ref{chapter:interoperation} except for Section~\ref{section:checking-bounds-interfaces},
Chapter~\ref{chapter:structure-bounds} except for Sections~\ref{section:member-bounds-state-extent}
and \ref{section:checking-bounds-with-structures},
and Chapter~\ref{chapter:simple-invariants} through the first section.  
Sections~\ref{section:pointer-cast-results},  \ref{section:integer-overflow-informal},
and \ref{section:pointer-integer-conversions} are advanced topics that can be read
after the other recommended chapters and sections.

We suggest language designers and compiler writers 
read most of the document. Language designers can skip Sections~\ref{section:computing-extent},
\ref{section:canonicalization}, and \ref{section:canonicalization-example}.
Compiler writers can skip the discussions in Section~\ref{section:programmer-dynamic-checks}
and Chapters~\ref{chapter:related-work} and \ref{chapter:design-alternatives}.

\section{Acknowledgements}

This design has benefited from many discussions with Weidong Cui, Gabriel Dos Reis, 
Chris Hawblitzel, Galen Hunt, Shuvendu Lahiri, and Reuben Olinsky.  The design has
benefited also from feedback from Sam Elliott, Michael Hicks, Wonsub Kim, Greg
Morrisett, Jonghyun Park, and Andrew Ruef. We thank them for their contributions to
the design.


