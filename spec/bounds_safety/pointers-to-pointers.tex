% !Tex root = checkedc.tex

\chapter{Pointers  to data with \plainarrayptr s}
\label{chapter:pointers-to-data-with-arrayptrs}

This chapter covers using pointers to data with \arrayptr s.  It also covers rules
around pointers and bounds expressions.  This includes dereferencing pointers
in bounds expressions and rules around address-taken pointer variables and
bounds expressions..
Three important scenarios for using pointers to \arrayptr s to support are:
\begin{itemize}
\item Data structures: All linked data structures in 
C are built with pointers, typically using pointers to structures.   This means
there need to be ways to create pointers to structures that contain \arrayptr s,
to assign through pointers to structures, and to read from pointers to structures.
\item Passing data by-reference to functions.  C does not provide by-reference
parameters.  Programmers pass data by-reference by using pointers explicitly.
To pass data of type \var{T} by reference, a programmer creates a parameter
with type \var{T *}.  Using checked pointers, a programmer create a parameter
with type \ptrT.
\item Taking the addresses of variables.  In C, programmers may take the
addresses of variables.  Typically, this is done to be pass data by-reference.
If a variable is used in a bounds expression, what can be done with the pointer
must be constrained so that bounds-safety is not accidentally broken.
\end{itemize}

\section{Initialization of pointers in data}

The term ``referent data'' will refer to the data that can be accessed
directly through a pointer.  The data may be an integer or floating-point number, a structure,
a pointer, or an array of data.   Any checked pointers in data must be initialized
before the pointers are used.

For variables, this can be enforced using a static 
analysis of variables with checked pointer types that point to data with checked
pointer types. For example, the Cyclone dialect of C \cite{Jim2002} did this
and languages like Java and C\# have rules for ensuring that variables
are definitely assigned before they are used.
For a variable of \arrayptr\ data, the compiler will insert code for 
initializing checked pointers in the data automatically to \texttt{0}, which is always
a valid pointer.    

For heap-allocated data, we can require that allocators
return data that has been zeroed.  The initial implementation of
Checked C will take this simple approach.  Section~\ref{section:allocation-of-uninitialized-data} 
describes a proposal for supporting for allocators that return pointers to
uninitialized data.
 
\section{Pointers to structures}

Pointers to  structures can be supported by ensuring that modifications to members
preserve type-level bounds invariants, following the approach of the Deputy
system \cite{Condit2007}.  This in turn means that accesses through structure
pointers return values that  follow the type-level bounds invariants, provided that modifications
are not interleaved.  Complicated aliasing problems are avoided because the
alias relationship of members within a structure is well-known.  Expressions
and statements that modify members must be relatively simple; they cannot access memory
that may be aliased.

Just like a member bounds can be suspended for a variable, a member bounds can 
be suspended for an object that is referenced by a pointer. The syntax is:

\begin{tabbing}
\var{fact:}\= \\
\>\texttt{suspends(\var{e}->\var{mp})} \\
\>\texttt{holds(\var{e}->\var{mp})} \\
\end{tabbing}

where \var{e} is a pointer to a structure type \var{S} and \var{mp} is a member of
path.  The syntax \texttt{(*\var{e}).\var{mp}} is equivalent to 
\texttt{\var{e}->\var{mp}} and can be used in its place.  
The syntax \texttt{suspends(*\var{e})} suspends all member bounds for 
the object pointed to by \var{e} and the syntax \texttt{holds(*\var{e})} 
means that all member bounds for \var{e} must hold.

The expression \var{e} must be a non-modifying expression to ensure that the \var{e}
refers to the same object at \keyword{suspends} and \keyword{holds} declarations.
It must meet the following additional restrictions:
\begin{itemize}
\item \var{e} cannot access memory via a pointer operation.
      It cannot contain a \texttt{*}, \texttt{->},  or a \texttt{[]} operation.
\item No variable whose value is used \var{e} can be modified between the \texttt{suspends}
and \texttt{holds} declarations.
\item No variable whose value is used in \var{e} can have its address taken. 
This is to avoid indirect modifications of variables.
\end{itemize}

There are additional restrictions to avoid problems due to pointer aliasing 
(when two pointers may point to the same memory).   We propose simple restrictions
for correctness.
Function calls are not allowed between \keyword{suspends} and 
\keyword{holds} declarations for \texttt{\var{e}->\var{mp}}.  The function calls 
might depend on the object pointed to by \var{e} having a  valid type-level invariant.
  
We continue to assume that the addresses of \arrayptr\ members and members
used in member bounds cannot be taken.  It is undefined behavior for unchecked
code to create pointers to the memory of those members.   With this assumption, we 
can allow assignments through and reads via other pointer expressions
between the \keyword{suspends} and \keyword{holds} declarations for
\texttt{\var{e}->\var{mp}}.  This is provided that the assignments do not modify 
\var{S.mp} or members in the bounds of \var{S.mp}.  No pointers should alias
\var{mp} or the other members, so the only aliases can be other structure
pointers.  Those can only be created through casting.  It is the programmer's responsibility
to update bounds-checked members properly in that case (Section~\ref{section:pointer-casting}).

The checking described in Chapter~\ref{chapter:structure-bounds} is extended from
operating over variable member paths to operating over expression member paths, where
the initial variable is replaced by a non-modifying expression that dereferences a structure
pointer.

\section{Speculative idea: support for by-reference parameter passing of \protect\arrayptr s}

It will be important to support by-reference parameter passing of \arrayptr-typed variables
in C.  It is a common way that multiple values are returned from a function.  By-reference parameter
passing of \arrayptr s will lead to programs where the addresses of local variables or struct
members of type \arrayptr\ are taken.

For example, a function that returns a newly-allocated buffer and its length could be written as:
\begin{lstlisting}
void create(ptr<array_ptr<char>> pbuf where *pbuf: count(*len),
            ptr<int> plen)
{ ...
}
\end{lstlisting}
A caller would take the addresses of local variables to use this function:
\begin{lstlisting}
void f(void)
{
    int len;
    array_ptr<char> buf : count(len) = NULL;
    create(&buf, &len);
}
\end{lstlisting}

We can gain insight into this problem by considering a related
problem.   When the address of a variable that occurs in a bounds declaration is 
taken, it is important to ensure that the invariants for the bounds declaration are not
broken accidentally.  Consider:
\begin{lstlisting}
void bad_code(array_ptr<char> buf : count(len), int len) 
{
    ptr<int> plen = &len;
    *plen = 5;
}
\end{lstlisting}
The problem with allowing the address of just \texttt{len} to be taken is
that the lvalue for \texttt{len} actually has additional constraints. This
suggests that pointer variables that hold addresses of address-taken variables should
themselves have additional constraints.  For example:
\begin{lstlisting}
void constrained_code(array_ptr<char> buf : count(len), int len) 
{
    ptr<int> plen = NULL;
    ptr<array_ptr<char>> pbuf = NULL where *pbuf : count(*plen);
    bundle {
       plen = &len;
       pbuf = &buf;
    }
    bundle {
       array_ptr<int> a = malloc((sizeof(char)*5);
       *plen = 5;
       *pbuf = a;
    }
}
\end{lstlisting}
There are some interesting issues that arise from this idea:
\begin{compactitem}
\item The bounds on the lvalues for \texttt{len} and
\texttt{buf} must be invariant throughout the execution of the program.
While we can put constraints on pointers that point to those lvalues,
we do not know when the lvalues will be updated.
The implication of this is that only local variables with invariant
bounds declarations (Section~\ref{section:invariant-bounds-declarations})
can have their addresses taken.   Local variables
with flow-sensitive bounds declarations cannot have their addresses taken.
The function could alter the bounds declarations for the local variables
so that they disagree with constraints on the pointers.
\item Once the address of a variable involved in a bounds declaration 
is taken, we cannot add new bounds declarations involving the variable.
The pointer-level bounds declarations would not reflect the new constraint.
\item The constraints on the pointers must fully capture 
all the constraints on the underlying variables.
\item Finally, constraints on the pointers cannot be dropped.
\end{compactitem}

The operations that could be done while updating
address-taken variables or assigning through pointers with 
constraints on them must be limited.
To maintain the bounds declaration, we must ensure that
variables are not modified unexpectedly via aliases. 
Without aliasing information, function calls could not
be allowed.  Updates to address-taken variables could
not be intermingled with reads or writes through pointers
with constraints to avoid the possibility of aliasing.

Taking the addresses of members that are used in member
bounds fits nicely into this approach for supporting
by-reference paramters.  The lvalues involved there do have
invariant bounds through the execution of the program,
which is the same requirement that we have placed on
local variables.

\section{Speculative idea: tracking initialization state of data}
\label{section:allocation-of-uninitialized-data}

In the long run, it is unsatisfactory to require that allocators always
return zeroed data.  Some allocators such as \texttt{malloc} return pointers to
uninitialized data.  This improves efficiency when data is
simply overwritten by the caller.  For example, it is not necessary
for bounds safety for arrays of characters to be zeroed.

Here we describe a strawman proposal for addressing this issue.  
We could add a way to track the initialization state of ranges of memory 
that contain pointers. We already have a way of 
describing ranges of memory: bounds expressions.  We can add
predicates that use bounds expressions to describe the initialization
state of memory pointed to by a variable:
\begin{itemize}
\item Three predicates can be used in \keyword{where}\ clauses
to describe the initialization state of a \arrayptr\ or \ptr-typed variable:
\keyword{\_init\_data}, \keyword{\_uninit\_data}, and \keyword{\_zero\_data}.
A predicate applies to a variable if it is combined using the \texttt{:}
notation: \texttt{\var{x} : \_uninit\_data} means that x points to data
that is uninitialized.
\item The predicates also have range versions that take bounds expressions
as arguments by suffixing the predicate name with \texttt{\_range}.  For example, 
if we want to express that the first n elements of \var{x} are initialized
and the remaining elements are not, we could have
\texttt{\var{x} : init\_data(x, x + n)
\&\& \var{x} : uninit\_data(x + n + 1, x + len)}, where \texttt{len} is 
the length of the array pointed to by \var{x}.
\item The default state of data is that it is assumed to be initialized.
Data must explicitly have another state declared for it to be in another
initialization state.
\item We need to address aliasing: static checking must ensure
that areas of memory do not have contradictary initialization states.
At a minimum, we would need a way to specify that a variable holds
a unique pointer to memory that no one else has \cite{Jim2002}.
\item There will be local rules for expression and statements that deduce that a new
initialization state declaration is valid after a statement.  We leave 
the rules to be worked out later.
\item Additional static checking rules will be added to ensure the
correctness with respect to bounds of casts to checked pointer types that have pointer types embedded
within them.
\end{itemize}

Here are potential bounds-safe interfaces for \texttt{malloc} and \texttt{calloc}.
We omit notation for describing that \texttt{malloc} and \texttt{calloc} produce 
pointers to new memory:
\begin{lstlisting}
void *malloc(size_t len) 
where return_value : byte_count(len) && return_value: _uninit_data;

void *calloc(size_t num, size_t size)
where return_value : byte_count(num * size)  && return_value : _zero_data;
\end{lstlisting}
A cast from the result of a \texttt{malloc} call to a checked pointer type with pointer types
would propagate the  \texttt{\_uninit} predicate:
\begin{lstlisting}
struct S {
    int len;
    int array_ptr<char> chars : len;
}

void f(void) 
{
   S *s = malloc(sizeof(S)) where s : _uninit_data;
   s->chars = NULL where s : _init_data;
   ...
\end{lstlisting}
Of course we might want inference rules that add annotations automatically.  

It might be possible to use ranges to write code that tracks at a fine grain that 
data is being zeroed:
\begin{lstlisting}
array_ptr<char> zero_mem(array_ptr<char> x : count(len) where x : _uninit, size_t len) 
   where x : _zero_init
{
   array_ptr<char> p : bounds(x, x + len) = x
       where p : _init_data(x, p) && p : _uninit_data(p, x + len);
       
   while (p < x + len) { 
       *p++ = 0;
   }
   return;
}
\end{lstlisting}