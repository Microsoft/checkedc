% !Tex root = checkedc.tex

\chapter{Extensions to existing C concepts}
\label{chapter:core-extensions}

This chapter describes extensions to existing C concepts. 
It covers new kinds of pointer types and array types, their meaning,
and the operations  on them.   It introduces two
new program scopes: \texttt{check} and \texttt{unchecked}
blocks. It covers a generalization of assertions to 
dynamic checks that are never  removed (unless a compiler
proves them redundant) because bounds safety depends upon them. 
Finally, it describes changes to
undefined behavior needed for bounds safety.

\section{New keywords}
Some of the extensions use new keywords.   Introducing new keywords
for C can cause conflicts with identifier names in existing C programs.
Fortunately, C provides a backward-compatible way to introduce new
keywords.  Identifiers that begin with an underscore followed by a
capital letter are reserved for system use, as are identifiers that
begin with two underscores \cite[Section 7.1.3]{ISO2011}. The following
new keywords are introduced:
\begin{verbatim}
_Array_ptr  _Checked  _Dynamic_check  _Ptr  _Where  _Unchecked
\end{verbatim}

It is desirable to have all-lowercase versions of the
identifiers for readability and ease of typing. The C
preprocessor is used to provide these. A standard header
file \keyword{stdcheckedc.h} is provided that has macros
that map the lowercase versions of keywords to the actual keywords.
Programs that do not have identifiers that conflict with the
lowercase versions of the keywords can include it.
Note that to allow header files to be included by programmers in any order,
the implementations of standard header files need to be modified to
not use identifiers that conflict with these keywords.
The all-lowercase versions of the keywords are:
\begin{verbatim}
array_ptr  checked  dynamic_check  ptr  where  unchecked
\end{verbatim}

The pattern of using an identifier reserved for system use coupled with
a header file was used before to introduce the boolean type. The keyword
name is \verb+_Bool+.  The standard header file \keyword{stdbool.h}
has a macro that maps \keyword{bool} to \verb+_Bool+.

Throughout this document, we use the shorter and easier-to-read
lowercase versions of the keywords.  It is assumed
that \keyword{checkedc.h} is included before examples.

\section{New kinds of pointer types}
Two new checked pointer types are added to C. Each pointer type can be
used in place of `\texttt{*}':
\begin{enumerate}
\item
  \ptrT: this points to
  a value of type \textit{T}. Pointer arithmetic is not allowed on these
  kinds of pointers. The expectation is that that many pointers are not
  involved in pointer arithmetic and will be given this type. When
  values of this pointer type are used to read or write memory, they
  must be non-null. The non-nullness is checked at runtime if necessary.
  The name \ptr\ is chosen for these kinds of pointers because these
  are expected to be the most common type of pointer in C code. \ptr\
  is a short succinct name.
\item
  \arrayptrT: this is
  a pointer to an element of an array of type \var{T} values. A
  variable of type \arrayptr\ that is used to access memory
  must have an associated declaration of the valid bounds for the
  variable. When values of this pointer type are used to read or write
  memory, they must be non-null and in bounds. The non-nullness and
  bounds are checked at runtime if necessary. Pointer arithmetic can be
  done on these pointer types. The resulting pointers do not need to be
  in bounds.
\end{enumerate}

Unchecked C pointer types that use `\texttt{*}' and are unchecked in their
ranges continue to exist. Pointer arithmetic is not forbidden on unchecked
pointer types because it would break existing C code. C compilers will
have an error or warning mode that flags unexpected uses of `\texttt{*}'.

The same syntax as C++ template instantiations is used for building
instances of these types because this syntax is well-known and
understood.  The new pointer types are added to the syntax for {\it type specifiers} 
\cite[Section 6.7.2]{ISO2011}. The parameters to these type constructors 
must be types, which are described syntactically by {\it type names}
\cite[Section 6.7.7]{ISO2011}.  If Checked C is extended to C++,
in the C++ extension these new types will be template types that have special meaning.

Checked pointers provide checking that memory accesses are in bounds.  They
do not provide checking that memory for objects is being managed
properly. For example, they can still be used to access memory for objects after
the objects have been deallocated.

Here are  examples of declarations using pointer types.   These declare variables that
are all pointers to integers:
\begin{verbatim}
int *p;
ptr<int> q;
array_ptr<int> s;
\end{verbatim}

The new checked pointer types can have \keyword{const} and \keyword{volatile}
modifiers, just like unchecked pointer types.   They can be applied to the type
of the object to which the pointers points or to the pointer itself.  Here are examples
of pointers to constant integers:
\begin{verbatim}
const int *p;
ptr<const int> q;
array_ptr<const int> r;
\end{verbatim}
A pointer to a constant integer cannot be used to modify the value of
the integer.

Here are examples of constant pointers to modifiable integers:
\begin{verbatim}
int x;
int *const p = &x;
const ptr<int> q = &x;
const array_ptr<int> r = &x;
\end{verbatim}
In this case, the pointers cannot be modified after they are defined, but 
the integer that they point to can be modified.   The checked pointer declarations 
have simpler syntax than the unchecked pointer declaration, where \keyword{const}
must be placed after the \verb+*+.   The checked pointer declarations are all analogous 
to \verb+const double y = 5.0;+ declaring a  variable \verb+y+ that is \keyword{const}.
In the declaration of \keyword{y}, \keyword{double} specifies a type.
The \keyword{const} keyword is placed before it to declare that
the variable in the declaration is \keyword{const}.  In the examples, \ptrint
and \arrayptrint\ all specify types too.

The checked pointer types follow the same rules that apply to modifiers for
unchecked pointer types.  For example, a variable that is a pointer to a
\keyword{const} value can be assigned a pointer to a non-const value.  
The reverse is not allowed.

\section{New kinds of array types}

A new checked array type is added to C. Just as there are checked pointer
types, there are checked array types. They are declared by placing the
modifier \keyword{checked} before the declaration of the bound of the
array\footnote{We can just as easily adopt the syntax that the checked
annotation is postfix and propagates from the inner most array to the
 outermost array. We have chosen the prefix syntax because the notation
 can be read easily from left to right ``this is a checked array of
 10x10 elements''.}:
\begin{verbatim}
int a checked[10];
\end{verbatim}

All array references to checked array types are bounds checked. C has
the rule that an ``array of \var{T}'' is converted implicitly to a
``pointer to \var{T}'' in a number of situations. This rule is extended
to convert a ``checked array of \var{T}'' to an ``\arrayptr\ 
to \var{T}''.

In C, array types may be complete or incomplete. A complete array type
specifies the bound of each dimension of the array using constant
expressions. An incomplete array type does not specify the bound of the
first dimension of the array. Examples of complete array types are
\texttt{int[10]} and \texttt{int[10][10]}. Examples of
incomplete array types are \texttt{int[]} and \texttt{int[][10]}.

If a checked array type is incomplete, there must be an associated
declaration of the valid bounds for the first dimension of the array.
For a complete array type, the bounds declared by the type are used to
bounds check array references. For example, given a declaration
\texttt{int a[10]} and a use \texttt{a[i]}, the bounds check is
that \texttt{i >= 0} and \texttt{i < 10}.

Array references to multi-dimensional arrays must be uniformly bounds
checked or not bounds checked. If any dimension is bounds checked, all
dimensions must be checked. A programmer can simply declare that
the outer dimension of an array is checked.  The checked property will be 
propagated to the inner dimensions:

\begin{verbatim}
int b checked[10][10];
\end{verbatim}

In C, multidimensional arrays are arrays of arrays,
where the nested array types have known dimensions at compile-time. A
2-dimensional array is an array of array of T, a 3-dimensional array is
an array of array of array of T. The checked property is propagated from the
outer array to the nested array types.

\subsection{An example}

It is easy to convert a function that operates only on complete array
types to one where array accesses are bounds checked: just add the
\keyword{checked} keyword to the declarations of the variables with complete array
types. Consider a method that adds 2x2 integer arrays \texttt{a} and
\texttt{b} so that \texttt{a} = \texttt{a} + \texttt{b}:

\begin{verbatim}
void add(int a checked[2][2], int b checked[2][2])
{
    for (int i = 0; i < 2; i++) {
        for (int j = 0; j < 2; j++) {
            a[i][j] += b[i][j];
        }
    }
}
\end{verbatim}

\subsection{Propagation of checked property to nested array types}
The checked property of an array type is propagated to a nested array type as follows.
A declaration of a variable has the form \var{T} \var{D},
where \var{T} is a type and \var{D} is a declarator. The declarator
can be as simple as an identifier \texttt{x}:
\begin{verbatim}
int x;
\end{verbatim}
It can be a more complex form that declares an identifier and modifies \var{T}
to produce a new type for the identifier. An example is \texttt{x[5]}:
\begin{verbatim}
int x[5];
\end{verbatim}

Given a declaration \var{T} \var{D}, if \var{D} is an array
declarator, it will have the form
\texttt{\var{D1}[\var{constant-expression\textsubscript{opt}}]},
where \var{D1} is another declarator. The type of the identifier in the
declaration \var{T D1} will be determined first. The type can be some
constructed type of the form \var{type-modifier} of \var{T}, where
\var{type-modifier} is a sequence of array, checked array, or pointer
type modifiers. If the first element in the \var{type-modifier}
sequence is an array or pointer, the type of the identifier will be
\var{type-modifier} ``array of  \var{T}''. If the first element in the
\var{type-modifier} sequence is a checked array, the type of the
identifier will be \var{type-modifier} ``checked array of \var{T}.''

For example, in parsing the declaration of \texttt{b}, \var{D1}
will be \texttt{int b checked[10]}. The type of \texttt{b} in
\var{D1} is ``checked array of int''. The type of \texttt{b} in
\texttt{int b checked[10][10]} will be ``checked array of
checked array of int''.

\subsection{Propagation and type definitions}

A \keyword{typedef} declaration defines a name for a type. This name
can be used to declare multi-dimensional arrays:
\begin{verbatim}
typedef int arr_ty[10];
arr_ty x[10];
\end{verbatim}
The checked property is propagated only to array declarators that are 
nested directly within other array declarators.   It is not propagated to
the bodies of type definitions.  It is an error if an array type and a nested 
array type from the use of a type name have different checked properties.

Here are examples of correct and incorrect declarations:
\begin{verbatim}
typedef int t1 checked[10];
t1 x checked[10];  // correct: checked properties matches

typedef int t2[10];
t2 x checked[10];  // error: mismatched checked properties
\end{verbatim}

\section{Operations involving pointer types}

The following operations involving pointer-typed values are allowed:

\begin{itemize}
\item
  Indirection: the \texttt{*} operator can be applied to a value of type
  \var{T} \texttt{*}, \ptrT, or \arrayptrT. It produces a value of type \var{T}.
\item
  Array reference: the \texttt{[]} operator can be applied to a
  value of type \var{T} \texttt{*} or \arrayptrT. It
  cannot be applied to a value of type \ptrT.
  \texttt{\var{e1}[\var{e2}]} is equivalent to
  \texttt{*(\var{e1} + \var{e2})}
\item
  Assignment: two pointers of the same type can be assigned.
\item
  Adding or subtracting a pointer type and an integer. This is allowed
  for \var{T} \texttt{*} and \arrayptrT types.
\item
  Pointers to objects of the same type can be compared for equality or
  inequality. The pointers do not have to be the same kind of pointer.
  To support reasoning about program behavior, the result of comparing
  pointers to different objects must be defined.  Checked pointers can also
  be compared for equality or inequality with 0 and void pointers, 
  just like unchecked pointers.
\item
  Pointers to objects of the same type can be compared relationally. Relational comparisons are the
  \verb|<|, \verb|<=|, \verb|>|, \verb|>=| operators. The pointers do not have to be
  the same kind of pointer. For example, an \uncheckedptrT\ can be compared with an
  \arrayptrT . To support bounds checking and reasoning about program behavior, the
  result of comparing pointers to different objects must be defined.
  Section~\ref{section:changes-to-undefined-behavior} describes this requirement in detail.
\item
  Pointers to objects of the same type can be subtracted. The pointers do not have to be
  the same kind of
  pointer. The result of subtracting pointers to different objects of
  the same type must be defined. Section~\ref{section:changes-to-undefined-behavior}
  describes this requirement in detail.
\end{itemize}

A value of one pointer type may be converted to a value of another
pointer type. For casts to or from checked pointer types where
bounds-safety can be checked at compile-time, a cast operator can be
used. If bounds safety cannot be checked at compile-time, bounds cast
operators must be used. There are two kinds of bounds cast operators:
\keyword{dynamic\_bounds\_cast}, which does runtime checks of any
conditions required to enforce bounds safety and
\texttt{assume\_bounds\_cast}, which does no runtime checking and
trusts the programmer. The rules for pointer casting and the bounds cast
operators are described in Section~\ref{section:pointer-casting}.

\subsection{Rules for the address-of operator}

If an address-of operator (\texttt{\&}) applied to an lvalue expression
of type \var{T}, the following rules apply:

\begin{itemize}
\item
  If the operator occurs in a checked block, the address-of operator
  produces a value of type \arrayptrT.
\item
  Otherwise, the address-of operator produces a value of type \var{T}
  \texttt{*}.
\end{itemize}

If the \texttt{\&} operator is applied to an lvalue expression that is a
dereference of a pointer or an array that has bounds checking, the
bounds check will be done when the address is taken. This guarantees
that it is valid to dereference the pointer that results from the
address-of operator. For example, in the following code, bounds checks
will be done at \texttt{\&a[i]} and \texttt{\&b[0]}.

\begin{verbatim}
int a checked[10];
array_ptr<int> b;
int i;
...
ptr<int> y = &a[i];
ptr<int> z = &b[0];
\end{verbatim}

\subsection{Rules for conversion of checked array types to pointer types}

If the type of an expression or a subexpression is ``checked array of
\var{T}'', the type of the expression is altered to be
\arrayptrT.

Following existing C language rules, this alteration does not happen if
the expression is an operand of an address-of operator, \texttt{++},
\texttt{--}, \texttt{sizeof}, or the left operand of an assignment
operator or the `\texttt{.}' operator. The prohibition against this
conversion occurring for operands of the address-of operator gives the
results of the operator a more precise type. For sizeof, it also results
in a more precise answer. The prohibitions against it for \texttt{++}, and
\texttt{--} operands and the left operand of an assignment operato keeps array
variables from being modifiable.

\subsection{Variable and member initialization}

Variables and members of variables with checked pointer types must be initialized
before they are used.  For \arrayptr s, this is checked as part of checking
the validity of bounds.  For \ptr s, this is checked
using a separate analysis.   A use includes taking the address of a variable
or member of a variable.  It is often unclear when the resulting pointer is used,
so the variable must be initialized when its address is taken.

\section{Pointer arithmetic error conditions}
\label{section:pointer-arithmetic-errors}

For existing unchecked C pointers, the definition of pointer arithmetic is
described in terms of addresses of elements of an array object. The C
Standard \cite{ISO2011} states, that given a pointer p that points to some element i of
an array object, p + j points to the (i+j)th element of that object.
Pointer arithmetic is defined only for pointers to elements of the array
object or one past the last element of the array object.

We take an alternative approach to defining the meaning of pointer
arithmetic and the error conditions for pointer arithmetic. Pointer
arithmetic is allowed to go out-of-bounds and has a well-defined
semantics. Section~\ref{section:new-pointer-types-semantics}
defines the semantics of the new pointer types
directly in terms of byte addresses, instead of with respect to
addresses of elements of an array. Pointer arithmetic that overflows or
involves a null pointer is defined to produce a runtime error.

The new pointer types allow pointer arithmetic that produces
out-of-bounds values. The C definition leaves pointer arithmetic that
produces out-of-bounds values undefined because it is not clear what the
meaning of should be when the pointers are dereferenced. The new pointer
types prevent out-of-bounds pointers from being dereferenced, and solve
this problem another way. In addition, in practice C implementations
often allow pointer arithmetic to produce out-of-bounds values and C
programs end up relying on this implementation-specific behavior. There
is no reason to cause existing code that computes out-of-bounds pointers
but does not dereference them to break when it is converted to use the
new pointer types.

When pointer arithmetic overflows or involves a null pointer, the
resulting value of the expression cannot be used and program execution
stops. If a system provides for error handling, an error handling
mechanism may be invoked to redirect program execution to a new point of
execution that does not use the value of the expression.

Defining pointer arithmetic this way simplifies reasoning about the new
pointer types. Expected identities such as \texttt{p + 1 > p} now hold
because, if \texttt{p + 1} overflows, the value cannot be
used. This allows programmers to narrow the bounds for
\arrayptr\ values by incrementing the lower bound or
decrementing the upper bound, even in situations where the bounds are at
the ends of the address space. Later sections describe places where
allowing an undefined value to be used would complicate reasoning about
programs.

If a compiler cannot prove that a new pointer type value is non-null
before a pointer arithmetic operation, it must insert a check.
Similarly, if a compiler cannot prove that a pointer arithmetic
expression for a new pointer type cannot overflow, it must insert a
check. This may slow a typical program down by a slight amount.

\subsection{Semantics of pointer arithmetic for new pointer types}
\label{section:new-pointer-types-semantics}

This section defines the semantics of pointer arithmetic and explains
when overflow occurs in pointer arithmetic. It is assumed that memory is
addressable at the byte level. The order of bits within a byte is not
specified. The order of bytes within built-in types larger than a byte,
such as integers and floating-point numbers, is also not specified.
Pointers shall be treated as addresses of locations of bytes in memory.
The addresses shall be unsigned integers with a defined range of 0 to
\texttt{UINTPTR\_MAX}. The maximum value of a signed integer that can be
added to an address shall be given by \texttt{INTPTR\_MAX} and the
minimum value of a signed integer that can be added to an address shall
be given by \texttt{INTPTR\_MIN}.

For the new
\arrayptrT\ pointer
types, there are distinct operations for addition and subtraction of
pointers by signed and unsigned integers. The operations behave
similarly, but have different overflow conditions for scaling because
the ranges of signed integers and unsigned integers are different.

\begin{itemize}
\item
  First scaling an integer by \sizeof{\var{T}} is
  defined. To scale an integer \var{i}, \var{i} shall be multiplied by
  \sizeof{\var{T}}, producing an integer \var{j}. If
  \var{i} is a signed integer, the scaled result shall be treated as a
  signed integer. If \var{i} is an unsigned integer, the scaled result
  shall be treated as an unsigned integer. For a signed integer, the
  minimum and maximum range for \var{j} shall be \texttt{INTPTR\_MIN} and
  \texttt{INTPTR\_MAX} For an unsigned integer, the minimum and maximum
  range for \var{j} shall be 0 and \texttt{UINTPTR\_MAX}. If \var{j} is
  outside its valid range, the operation doing the scaling operation
  shall produce a runtime error.
\item
  \var{p} \texttt{+} \var{i}, where \var{p} is an
  \arrayptrT\ pointer
  and \var{i} is an integer. The integer \var{i} shall be scaled by
  \sizeof{\var{T}}, producing an integer \var{j}. The
  pointer \var{p} will be interpreted as an unsigned integer. The mathematical
  value \var{p} + \var{j} shall be the result of the operation. If
  \var{p} + \var{j} is out of range for a pointer, the operation shall
  produce a runtime error.
\item
  \var{i} \texttt{+} \var{p}, where \var{p} is an
  \arrayptrT\ pointer
  and \var{i} is an integer, shall be defined as \var{p} \texttt{+}
  \var{i}.
\item
  \var{p} \texttt{-} \var{i}, where \var{p} is an
  \arrayptrT\ pointer
  and \var{i} is an integer. The integer \var{i} shall be scaled by
  \sizeof{\var{T}}, producing an integer \var{j}. The
  pointer \var{p} will be interpreted as an unsigned integer. The
  mathematical value \var{p} - \var{j} shall be the result of the
  operation. If \var{p -} \var{j} is out of range for a pointer, the
  operation shall produce a runtime error.
\item
  \var{p} \texttt{-} \var{q}, where \var{p} and \var{q} are
  \arrayptrT\
  pointers. The two pointers will be interpreted as unsigned integers
  and the mathematical value \var{p} - \var{q} shall be computed,
  producing an integer \var{j}. If \var{j} is out of range for signed
  integer that can be added to an address, the operation shall produce a
  runtime error. If \var{j} is a multiple of
  \sizeof{\var{T}}, the result shall be
  \var{j}/\sizeof{\var{T}}. If \var{j} is not a
  multiple of \sizeof{\var{T}}, then the value shall
  be determined as follows:

  \begin{itemize}
  \item
    If \var{j} is non-negative,
    \var{j}/\sizeof{\var{T}} shall round toward 0.
  \item
    If \var{j} is negative, it shall be implementation-defined whether
    \var{j}/\sizeof{\var{T}} rounds toward 0 or away
    from 0.
  \end{itemize}
\end{itemize}

An important implication of these definitions is that they put a maximum
limit on the number of elements in an array of type \var{T}. It is
\texttt{UINTPTR\_MAX}/\sizeof{\var{T}}. They also put
maximum limits on the number of elements that can be accessed in an
array by a signed integer or an unsigned integer. That leads to limits
on the size of arrays that can be described by some bounds. A signed
integer that must be non-negative can describe an array of
\texttt{INTPTR\_MAX/\sizeof{\var{T}}} elements.

\subsection{Expressing pointer arithmetic as integer arithmetic}
\label{section:pointers-as-integers}

During static checking, pointer arithmetic operations will be converted
to use integer arithmetic. This is necessary in C because at times
programmers do explicit size computations that follow the same rules as
pointer arithmetic.

To support this expansion, integer arithmetic operators are extended
with the operators \texttt{+\textsubscript{ovf}},
\texttt{-\textsubscript{ovf}} , and \texttt{*\textsubscript{ovf}}. The
operators interpet pointers as unsigned integers in some range 0 to
\texttt{UINTPTR\_MAX}. An operator produces a runtime error if the value
of its result cannot be represented by the result type:

\begin{itemize}
\item
  \texttt{+\textsubscript{ovf}} takes an unsigned integer \var{i} and an
  integer \var{j} and produces an unsigned integer in the range 0 to
  \texttt{UINTPTR\_MAX}. Its result is the mathemetical value ]var{i} + \var{j}.
\item
  For subtraction, there are two forms:

  \begin{itemize}
  \item
    \texttt{-\textsubscript{ovf}} takes an unsigned integer \var{i} and an
    integer \var{j} as an argument and computes \var{i} - \var{j}, producing an unsigned
    integer in the range 0 to \texttt{UINTPTR\_MAX}. Its result is the
    mathematical value \var{i} - \var{j}.
  \item
    \texttt{-\textsubscript{ovf\_diff }}takes two unsigned integers \var{i}
    and \var{j} and computes \var{i} - \var{j}, producing a signed integer of type
    \texttt{ptrdiff\_t}. Its result is the mathemetical value \var{i} - \var{j}.
  \end{itemize}
\item
  \texttt{*\textsubscript{ovf }}takes two integers \var{i} and \var{j} (both either
  signed or unsigned) as arguments. It produces an integer whose type is
  the same as the input argument types. Its result is the mathematical
  value \var{i} * \var{j}.
\end{itemize}

Given an expression \texttt{e1} with a pointer type and an expression
\texttt{e2} with an integer type, the expansion of \texttt{e1 + e2} from
pointer arithmetic to integer arithmetic depends on the type of
\texttt{e2}. The number of bytes to added must be the same kind of
signed or unsigned integer as \texttt{e2}.

\begin{itemize}
\item
  If \texttt{e2} is an unsigned integer, \texttt{e1 + e2} expands to
  \texttt{e1 +\textsubscript{ovf} \sizeof{\var{T}} *\textsubscript{ovf} e2}
\item
  If \texttt{e2} is a signed integer, the expansion of \texttt{e1 + e2}
  must cast \sizeof{\var{T}} to a signed integer. We introduce a
  signed integer type \texttt{signed\_size\_t} that is large enough for
  this cast. \texttt{e1 + e2} expands to \texttt{e1 +\textsubscript{ovf}
  ((signed\_size\_t) \sizeof{\var{T}}) *\textsubscript{ovf} e2}. This cast is
  necessary because in C, multiplying a signed integer by an unsigned
  integer implicitly converts the signed integer to be an unsigned
  integer.
\end{itemize}

\subsection{Runtime performance implications}

There will be concerns about the effect of overflow checks on the speed
of pointer arithmetic using the new pointer types. These concerns are an
empirical question to be settled after implementing and using the new
pointer types. It is unclear what the actual cost will be. First, there
will be sometimes be additional conditions on expressions used in
pointer arithmetic that prevent overflow from occurring. Second,
compiler optimizations often can remove the checks. Third, programmers
can use lightweight invariants to show statically that checks are not
necessary.

In our experience working on an operating system written in managed code
that required these checks, the slowdown was a few percent, even when
\emph{all} arithmetic instructions were checked for overflow too. The
cost of checks was low because it is often the case that checks are
redundant or can be combined into adjacent machine instructions. For
example, in the code \texttt{t = *p; p += 1;} the first dereference of
\texttt{p} implies that \texttt{p} is non-null before the increment.
Otherwise, in a typical C environment, the program would fault at
\texttt{*p}. In the code \verb|if (p < hi) { p += 1; }|
the check \texttt{p < hi} implies that the increment cannot
overflow. The checks are also inexpensive on superscalar processors:
they can be placed so that they are statically predicted by 
processors to never fail.

\section{Relative alignment of \arrayptr\ values}
\label{section:relative-alignment}

\arrayptrT\ pointers provide
pointer arithmetic on arrays. The bounds for these pointers usually
describe a range of memory that is exactly the size of some array of \var{T}.
The pointers point to an element of the array. In other words, the lower
bound, the upper bound, and pointer are  relatively aligned to the type
\var{T}. Given a lower bound \var{lb}, an upper bound \var{ub}, and a
pointer \var{p}, there should exist some integer \var{count} such that
\var{ub} = \var{lb} + \var{count}. In addition, there is either some
integer \var{index} such that \var{p} = \var{lb} + \var{index},
where addition is C pointer arithmetic, or \var{p} is null.

The type to which a pointer and its bounds are relatively aligned is
called the relative alignment type. By default, the relative alignment
type for a pointer and its bounds is the referent type. However, the
relative alignment type can be overridden by specifying it explicitly as
part of the bounds.  This is described in 
Section~\ref{section:representing-relative-alignment}.
This can be used for bounds for the results of pointers casts and 
for  \arrayptrvoid\ pointers. The type
\texttt{void} has no defined size, so the default relative alignment is
undefined for \texttt{void}.

We considered removing the entire concept of relative alignment from the
design to simplify the design.  We decided against that because it would 
increase the cost of bounds checking throughout the program.  
Section~\ref{section:design-alternatives:always-unaligned} explains
this choice in more detail.

\section{Program scopes for checked pointer types}

To improve program reliability and to simplify understanding programs,
it is desirable to limit code to using only checked pointers. To support
this, we introduce checked program scopes. The \keyword{checked} keyword
can be attached to blocks and function definitions. In checked program
scopes, the declared types of variables are allowed to be or use
checked pointer types or checked array types; unchecked pointer types
and unchecked array types are not allowed.  Similarly, for functions,
return types and parameter types are allowed to be or use checked pointer
types or checked array types.  

On the other hand,  declarations in checked scopes can use unchecked pointer
types and unchecked array types, provided that the declarations provide a
bounds-safe interface.   These are described in
Section~\ref{section:function-bounds-safe-interfaces}.
Variables and functions used in checked scopes are 
allowed to have or use checked pointer types, checked array types, or
unchecked pointer types and unchecked array types with bounds declarations.

A new pragma directive \texttt{BOUNDS\_CHECKED} is introduced to control whether
the top-level scope is a checked scope:
\begin{alltt}
#pragma BOUNDS_CHECKED \textit{on-off-switch}
\end{alltt}

Where \texttt{\textit{on-off-switch}} is one of \verb|on off|
and it is \verb|off| by default

By default, function definitions are not checked. A block inherits the
checking properties of its parent. This preserves the meaning of
existing C code.

A checked block is introduced by placing  \keyword{checked} before a
compound block:
\begin{verbatim}
checked 
{
    int a = 5;
    ptr<int> pa = &a;

    int b checked[5][5];
    for (int i = 0; i < 5; i++) {
        for (int j = 0; j < 5; j++) {
            // all references are bounds checked
            b[i][j] = -1;
        }
    }
}
\end{verbatim}

It is rarer for a programmer to need to introduce an unchecked scope. It
is needed usually to allow the use of unchecked pointers within a checked
block. The \keyword{unchecked} keyword can be used in all the places where the
checked keyword can be used. This example shows the use of an unchecked
block:

\begin{verbatim}
checked 
{
    int a = 5;
    unchecked 
    {
        int *upa = &a;	
        int b[5][5];
        for (int i = 0; i < 5; i++) {
            for (int j = 0; j < 5; j++) {
                // not bounds checked
                b[i][j] = -1;
            }
        }     
    }
 ...
}
\end{verbatim}

In a checked function definition, the body of the function is a 
checked  block. A checked function definition is declared by placing the
\keyword{checked} keyword before the definition. Here are examples of checked and
unchecked function definitions:

\begin{verbatim}
// checked at the function level: no unchecked pointers can appear in argument
// types, the return type, or the body of the function.
checked int f(ptr<int> p) 
{
    int a = 5;
    ptr<int> pa = &a;
    ...
}

// unchecked at the function level: checked and unchecked pointer types can
// occur in argument types, the return type, or the body of the function
unchecked int f(int *p, ptr<int> r)
{
    int a = 5;
    int *pa = &a;
    ...
}

// f is unchecked by default
int f(int *p, ptr<int> r)
{
    int a = 5;
    int *pa = &a;
    ...
}
\end{verbatim}

When a function call occurs in a checked block, the function being
called does not have to be declared as checked. The notion of whether a
scope is checked or not checked is lexical and the function definition
is a separate lexical scope.

C allows declarations of functions without prototypes, where the types 
of the arguments to functions are not specified.  These
functions are dangerous to use because there can be mismatches 
between argument types and parameter types at function
calls.  This can corrupt data or the call stack.  In checked scopes, 
the use or declaration of  functions without prototypes is not allowed.

As we add different notions of checking to Checked C, we will use the
checked and unchecked keywords for all the different notions of
checking. We may introduce additional keywords to control specific kinds
of checking.

\section{Programmer-inserted dynamic checks}
\label{section:programmer-dynamic-checks}

A bounds check generates a runtime error if it fails. The ability to
generate a runtime error is not limited to the C implementation. A
programmer can check a Boolean condition and generate a runtime error
using the expression \texttt{dynamic\_check(}\var{e1}\texttt{)}, where
\var{e1} is an integral valued expression. At runtime, \var{e1} is
evaluated. If the result is \texttt{0} (false), a runtime error occurs.
If the result is non-zero (true), no runtime error occurs. Just as with
a bounds check, if a runtime error occurs and a C implementation
provides an error-handling facility, the error-handling facility may be
invoked.

The \texttt{dynamic\_check} expression is similar to an assertion, but
unlike an assertion, it is expected to be used in production or release
versions of software. The \texttt{dynamic\_check} expression is useful
for these reasons:

\begin{itemize}
\item
  It provides an escape hatch for limits of static checking. If a
  programmer knows a condition is true at runtime, yet the static
  checking cannot prove the fact, the programmer can use
  \texttt{dynamic\_check} to show that the condition is true.
\item
  It maintains programmer control: programmers can use unchecked pointers
  and \texttt{dynamic\_check} to write the same code that the compiler
  would generate for checked pointers.
\item
  It gives programmers more control over bounds checks. A programmer can
  place a pre-condition before a loop that ensures that the loop is free
  from dynamic bounds checks, without having to restructure the
  control-flow of the program.
\end{itemize}

The following example illustrates why having an escape hatch from static
checking is useful. Suppose a decoder from a compressed representation
to an uncompressed representation is being converted to use checked
pointers. This example is based on code patterns seen in the Abstract
Syntax Notation (ASN1) parsing code of OpenSSL \cite{OpenSSL2015}.

\begin{alltt}
void decode(char *output\_buffer, char *input\_buffer, size\_t input\_len)
\{
    char *src = input\_buffer;
    char *src\_bound = src + input\_len;
    char *dst = out\_buffer;

    while (src < src\_bound) \{
        switch (*current++) \{
            case UNCOMPRESSED\_BYTES: \{ 
                // just copy bytes; compression wasn't useful here.
                size_t len = src[0] + src[1]*256;
                src += 2;
                memcpy(dst, src, len);
                src += len;
                dst += len;
                break;
            \}
            case COMPRESSED\_INT64: \{
                ...
                break;
            \}
        ...
    \}
\}
\end{alltt}

The caller knows that the destination buffer will be large enough and
that the contents of the source buffer are well-formed. However, these
invariants cannot be expressed using lightweight invariants. These are
complicated high-level invariants that require the use of techniques for
proving functional correctness.

To use checked pointers, the size of the destination buffer must be passed
in and there must be a check before the memcpy that the destination
buffer and source buffer have enough room. We ignore the details of how
the bounds are described for now.    The new code for checked pointers is italicized
and highlighted in yellow:

\begin{alltt}
void decode(array\_ptr<char> output\_buffer, array\_ptr<char> input\_buffer, 
            size\_t input_len\hl{\textit{, size\_t output_len}})
\{
    array\_ptr<char> src = input\_buffer;
    array\_ptr<char> src\_bound = src + input\_len;
    array\_ptr<char> dst = out\_buffer;
    \textit{array\_ptr<char> dst\_bound = out\_buffer + output\_len;}


    while (src < src\_bound) \{
        switch (*current++) \{
            case UNCOMPRESSED\_BYTES: \{ 
                size\_t len = src[0] + src[1]*256;
                src += 2;
                \hl{\textit{// need check that dst and src have at least len bytes of}}
                \hl{\textit{// space}}
                memcpy(dst, src, len);
                src += len;
                dst += len;                
                break;
            \}
            case COMPRESSED\_INT64: \{
                ...
                break;
            \}
        ...
    \}
\}
\end{alltt}

How should that check be written? One approach is to change the
control-flow by inserting if-statements into the program. Something must
be done if the check fails, though. One possibility is to just ignore a
failure. This is a bad programming practice because now the program might
fail silently:

\begin{alltt}
void decode(array\_ptr<char> output\_buffer, array\_ptr<char> input\_buffer, 
            size\_t input\_len\textit{, size\_t output\_len})
\{
      ...
            case UNCOMPRESSED\_BYTES: \{
                size\_t len = src[0] + src[1]*256;
                src += 2;
                \hl{\textit{// check that dst and src have at least len bytes of}}
                \hl{\textit{// space}}
                \hl{\textit{if (dst + len >= dst\_bound || src + len >= src\_bound) \{}}
                \hl{\textit{   goto failure;}}
                \hl{\textit{\}}}
                memcpy(dst, src, len);
                src += len;
                dst += len;                
                break;
            \}
   ...
   \textit{failure:}
   \textit{   return;}
\}
\end{alltt}

This could be fixed by having \texttt{decode} return a status code
indicating success or failure. That just pushes the problem upward to
the caller and leaves a testing problem. The program should never fail,
so there is no way to test the path.

\begin{alltt}
\textit{int} decode(array\_ptr<char> output\_buffer, array\_ptr<char> input\_buffer, 
            size\_t input\_len\hl{\textit{, size\_t output\_len}})
\{
   ...
\hl{\textit{   failure:}}
\hl{\textit{      return 1;}}
\}
\end{alltt}

The problems with requiring functions that validate buffer lengths to
return status codes for errors are analyzed by O'Donell and Sebor\cite{ODonell2015}. 
Annex K of the C Standard \cite{ISO2011} introduced a new set of standard library functions to replace
functions that provide no way to validate their arguments. These
functions return status codes to indicate success or failure A classic
example of a function prone to misuse is \texttt{strcpy(char *dst, const
char *src)}. It copies all bytes in \texttt{src} to \texttt{dst} until
it hits a null byte. If \texttt{src} is missing the null byte or
\texttt{dst} is too small, this causes a buffer overrunn. The new
function \texttt{strcpy\_s} takes an additional size parameter for
\texttt{dst} and has the signature \texttt{errno\_t strcpy\_s(char *dst,
size\_t dest\_len, const char *src)}. O'Donell and Sebor explain how
using these functions is awkward, leading to more complicated and less
efficient code.

In contrast, \texttt{dynamic\_check} allows the checking to be localized
and not propagate upward in the call chain. If the programmer is
correct, the check never fails. If the programmer is incorrect, the
check might fail and invoke error-handling code:

\begin{alltt}
void decode(array\_ptr<char> output\_buffer, array\_ptr<char> input\_buffer, 
            size\_t input\_len\textit{, size\_t output\_len})
{
      ...
            case UNCOMPRESSED\_BYTES: { 
                size\_t len = src[0] + src[1]*256;
                src += 2;
                \hl{\textit{dynamic\_check(dst + len < dst\_bound && src + len < src_bound);}}
                memcpy(dst, src, len);
                src += len;
                dst += len;                
                break;
            }
   ...
}
\end{alltt}

One can argue that it is a problem to have a dynamic point of failure
that leads to error-handling code being invoked. This is the same way
systems treat null pointer dereferences, though, which are a possibility
throughout C code. The alternative of having a program with undefined
behavior is worse.

The following example uses \texttt{dynamic\_check} to eliminate bounds
checks in a loop. It is based on experience hand-optimizing C\# and Java
programs. This kind of example is typically found during a performance
tuning phase of program development. In the example, 
the \verb|: count(|\var{exp}\verb|)| notation indicates that \var{exp} is the
length of the buffer.

\begin{verbatim}
void append(array_ptr<char> dst : count(dst_count),
            array_ptr<char> src : count(src_count), 
            size_t dst_count, size_t src_count)
{ 
    for (size_t i = 0; i < src_count; i++) {
        if (src[i] == marker) {
           break;
        }
        dst[i] = src[i];
    }
}
\end{verbatim}

The italicized expressions are bounds checked:
\begin{alltt}
void append(array\_ptr<char> dst: count(dest\_count), 
            array\_ptr<char> src : count(src\_count), 
            size\_t dst\_count, size\_t src\_count)
\{ 
    for (size\_t i = 0; i < src\_count; i++) \{
        if (\hl{\textit{src[i]}} == marker) \{
           break;
        \}
        \hl{\textit{dst[i]}} = \hl{\textit{src[i]}};
    \}
\}
\end{alltt}

It is clear that the accesses to \texttt{src} are in-bounds based on
just information from the for-loop, so a compiler will eliminate those
bounds checks. It is not clear that assignments through \texttt{dst} are
always in bounds, so the check there must remain. It can be eliminated
by adding a \texttt{dynamic\_check}:

\begin{alltt}
void append(array\_ptr<char> dst: count(dest\_count), 
            array\_ptr<char> src : count(src\_count), 
            size\_t dst\_count, size\_t src\_count)
\{ 
    \hl{\textit{dynamic\_check(src\_count <= dst\_count);}}
    for (size\_t i = 0; i < src\_count; i++) \{
        if (src[i] == marker) \{
            break;
        \}
        dst[i] = src[i];
    \}
\}
\end{alltt}

The compiler now knows that \texttt{i < src\_count
<= dst\_count}, so it can eliminate the check.

A compiler would not introduce this dynamic\_check because it would
alter the behavior of \texttt{append}. The bounds check on \texttt{dst}
in the original code is done only if a marker is not found in
\texttt{src} and \texttt{src\_count \textgreater{} 0}. A compiler could
try to deduce a precondition for the loop 
that prevents the bounds check from failing, but this is
not possible because the precondition depends on the contents of
\texttt{src}. A compiler would have to clone code to maintain the same
behavior. This increases code size, so production compilers do not this
sort of transformation or do it sparingly. Programmer control produces
better results.   Here is the code a compiler might generate with cloning.

\begin{alltt}
void append(array\_ptr<char> dst: count(dest\_count), 
            array\_ptr<char> src : count(src\_count), 
            size\_t dst\_count, size_t src\_count)
\{ 
    if (src\_count <= dst\_count) \{
        for (size\_t i = 0; i < src\_count; i++) \{
            if (src[i] == marker) \{
                break;
            \}
            dst[i] = src[i];  // no bounds check needed
        \}   
    else \{
        for (size\_t i = 0; i < src\_count; i++) \{
            if (src[i] == marker) \{
                break;
            \}
            \hl{\textit{dst[i]}} = src[i];  // bounds check needed
        \}   
    \}
\}
\end{alltt}

\section{Changes to undefined behavior}
\label{section:changes-to-undefined-behavior}

C has situations where an expression has undefined behavior or the
meaning of an expression is undefined:

\begin{itemize}
\item
  Pointer arithmetic for unchecked pointer types only has defined behavior when
  the resulting pointer points to the same object as the original pointer, or one
  element past the object.
\item
  Pointer comparison for unchecked pointer types only has defined behavior when comparing
  pointers to the same object (where one or both pointers may point one
  element past the same object).
\item
  Pointer subtraction for unchecked pointer types only has defined behavior when subtracting
  pointers to the same object (where one or both pointers may point one
  element past the object).
\item
  Arithmetic expression behavior is undefined on signed integer overflow
  and integer division by 0.
\item
  Expressions may have nested assignments within them. The evaluation
  order of side-effects in subexpressions is defined only in specific
  circumstances; otherwise it is undefined. This leads to expressions
  with undefined meaning. There can be multiple assignments to the same
  variable that have no defined evaluation order with respect to each
  other or an assignment and a use of a variable that have no defined
  evaluation order.
\item
  Initializers may have nested assignments within them. These can have
  undefined meanings as well for the same reasons as expressions.
\end{itemize}

Undefined behavior is different from unspecified behavior, where one of
a number of choices may be made. Unspecified behavior in C includes:

\begin{itemize}
\item
  The order of evaluation of side-effects in expressions (so long as the
  expression does not have undefined behavior).
\item
  The order of evaluation of side-effects in initializers.
\end{itemize}

C also has implementation-defined behavior, which includes:

\begin{itemize}
\item
  The ranges of values for integer, floating-point, and pointer types.
\item
  Data layout, including the sizes of types, padding, and alignment of
  data.
\item
  Some aspects of pointer conversion.
\end{itemize}

It is difficult to reason about the correctness of programs that have
expressions with undefined behavior. One has to make sure that a program
never arrives at a point where behavior is undefined. In practice, this
would mean proving that signed integer overflow can never occur, for
example. For unspecified behavior, one has to reason about all possible
behaviors, such as all possible orders of evaluation. For
implementation-defined behavior, one has to reason about the
implementation-specific behavior or have reasoning that is independent
of the details.

A careful reading of the rules for unchecked pointer comparison implies
that it is impossible to detect an out-of-bounds unchecked pointer in C,
for example. If an unchecked pointer \var{p} is not in the valid range for an
object, the pointer comparison is undefined.

To provide for pointer bounds safety, we require that C implementations
provide defined behaviors for unchecked pointer arithmetic operations and
signed integer overflow:

\begin{itemize}
\item
  Unchecked pointers shall be treated as addresses of locations in memory,
  just as checked pointers are treated as addressses. The addresses shall
  be unsigned integers with a defined range of 0 to
  \texttt{UINTPTR\_MAX}:

  \begin{itemize}
  \item
    Comparison of pointers for all different kinds of pointers shall be
    defined as the corresponding integer comparison.
  \item
    Subtraction \var{p} \texttt{-} \var{r} of two pointers \var{p} and \var{r}
    of type \var{T} where one
    pointer is a checked pointer and the other is an unchecked pointer shall
    be done following the rules for subtraction of checked pointers,
    treating the unchecked pointer as a checked pointer in those rules.
  \end{itemize}
\end{itemize}

\begin{itemize}
\item
  To be able to maintain pointer bounds safety, it is important that
  signed integer overflow produce a defined value. When a signed integer
  expression produces an out-of-range value, either (1) the operation
  must convert that value to an in-range integer value or (2) the
  expression shall produce a runtime error. The conversion must be a
  function of only the input values of the expression.
\item
  Integer division by 0 shall also produce a runtime error or produce a
  defined value.
\end{itemize}

In the case of a runtime error, program execution cannot continue in a
way the uses the value of the expression that produced the error.

For programs with expressions and initializers with undefined meanings,
those programs must be rejected at translation-time. 
Section~\ref{section:avoiding-undefinedness}
describes this in detail.

Of course, there are other ways in which C expressions may have
undefined behavior:

\begin{enumerate}
\item
  By reading or writing memory through an out-of-bounds pointer.
\item
  By storing a value of one type and accessing the value as a different
  type later, when the value is not valid for the different type. A
  program might write bit patterns that do not correspond to valid
  values for a type or write an integer and use it as a pointer later,
  even though the pointer is not within the range of memory for a valid
  object.
\item
  By accessing memory that has been deallocated.
\item
  By using variables or functions with inconsistent declarations across
  different translation units. This can cause the type of a variable or
  a function to be different at the definition and the use. This can be
  addressed with suitable link-time checking.
\end{enumerate}

The aim is to be able to show partial correctness of programs for the
first item (avoiding out-of-bounds pointer accesses). A
partial correctness guarantee has the form ``assuming X holds, then Y is
true''. Informally, one might say ``assuming that memory allocation and
type casts are correct and that unchecked code never reads or write though
out-of-bounds pointers, then checked code never reads or writes through
out-of-bounds pointers.'' These assumptions can be turned into formal
statements about program behavior at runtime. Given those assumptions,
we might then prove that at runtime checked code never reads or writes
through out-of-bounds pointers.
